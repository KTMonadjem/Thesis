
\usepackage[dvipsnames,table]{xcolor}

\usepackage{pgf}
\usepackage{pgfplots,pgfplotstable}
\usepackage{tikz}
\usepackage{tikzscale}
\usetikzlibrary{shapes,backgrounds}
\usetikzlibrary{arrows,shapes,fit,automata,positioning,decorations,calc}
\usetikzlibrary{spy,backgrounds}
\usetikzlibrary{arrows.meta}
\usepackage{siunitx}
\pgfplotsset{compat=1.12}
\usepackage{graphicx}
\usepackage{amsmath}
\usepackage{xspace}
\usepackage{pgfplots}


\usepackage{bbold}
\usepackage{pslatex} % -- times instead of computer modern, especially for the plain article class
%\usepackage[colorlinks=false,bookmarks=false]{hyperref}
\usepackage{booktabs}

\usepackage{graphicx}
\usepackage{xcolor}
\usepackage{multirow}
%\usepackage{cite}
\usepackage[normalem]{ulem} %for striking out text
%
\usepackage{amsmath}
\usepackage{amssymb}
\usepackage{amsfonts} %
\usepackage{calligra} %
\usepackage{mathtools}

\usepackage{amsthm} % for neater definitions

\usepackage{dblfloatfix} %allows floats to be at bottom of page

%tikz and associated stuff
\usepackage{verbatim}
\usetikzlibrary{shapes,backgrounds}
\usetikzlibrary{arrows,fit,automata,positioning,decorations,calc}
\usetikzlibrary{spy}
\usetikzlibrary{matrix,chains,decorations.pathreplacing}
%\usetikzlibrary{arrows.meta}

\usepackage{caption}
\usepackage{subcaption}

\usepackage{comment}
\usepackage{listings}

\usepackage{pgfplots}
\usepackage{pgfplotstable}

\usetikzlibrary{patterns}

\usepackage[percent]{overpic}

\usepackage[english]{babel}
\usepackage{blindtext}

\usepackage{acronym}

%\usepackage[subtle]{savetrees}

%\usepackage{flushend} % even out the last page, but use only at the end when there is a bibliography

\newcommand\defeq{\stackrel{\mathclap{\scriptsize\mbox{def}}}{=}}
\newcommand{\code}[1]{{\small{\texttt{#1}}}}

% fatter TT font
\renewcommand*\ttdefault{txtt}
% another TT, suggested by Alex
% \usepackage{inconsolata}
% \usepackage[T1]{fontenc} % needed as well?

\usepackage{listings}
\usepackage{float}
\definecolor{mygreen}{rgb}{0,0.6,0}
\definecolor{mygray}{rgb}{0.5,0.5,0.5}
\definecolor{mymauve}{rgb}{0.58,0,0.82}

\lstset{ 
  backgroundcolor=\color{white},   % choose the background color; you must add \usepackage{color} or \usepackage{xcolor}; should come as last argument
  basicstyle=\footnotesize,        % the size of the fonts that are used for the code
  breakatwhitespace=false,         % sets if automatic breaks should only happen at whitespace
  breaklines=true,                 % sets automatic line breaking
  captionpos=b,                    % sets the caption-position to bottom
  commentstyle=\color{mygreen},    % comment style
  deletekeywords={...},            % if you want to delete keywords from the given language
  escapeinside={\%*}{*)},          % if you want to add LaTeX within your code
  extendedchars=true,              % lets you use non-ASCII characters; for 8-bits encodings only, does not work with UTF-8
  frame=single,	                   % adds a frame around the code
  keepspaces=true,                 % keeps spaces in text, useful for keeping indentation of code (possibly needs columns=flexible)
  keywordstyle=\color{blue},       % keyword style
%  language=Esterel,                 % the language of the code
  morekeywords={*,...},            % if you want to add more keywords to the set
  numbers=left,                    % where to put the line-numbers; possible values are (none, left, right)
  numbersep=5pt,                   % how far the line-numbers are from the code
  numberstyle=\tiny\color{mygray}, % the style that is used for the line-numbers
  rulecolor=\color{black},         % if not set, the frame-color may be changed on line-breaks within not-black text (e.g. comments (green here))
  showspaces=false,                % show spaces everywhere adding particular underscores; it overrides 'showstringspaces'
  showstringspaces=false,          % underline spaces within strings only
  showtabs=false,                  % show tabs within strings adding particular underscores
  stepnumber=1,                    % the step between two line-numbers. If it's 1, each line will be numbered
  stringstyle=\color{mymauve},     % string literal style
  tabsize=2,	                   % sets default tabsize to 2 spaces
  title=\lstname                   % show the filename of files included with \lstinputlisting; also try caption instead of title
}
\newcommand{\todo}[1]{{\color{orange}(TODO: #1)}}
\newcommand{\hammond}[1]{{\color{blue} Hammond: #1}}
\newcommand{\matthew}[1]{{\color{red} Matthew: #1}}
\newcommand{\changed}[1]{{\color{red}#1}}

\theoremstyle{definition}
\newtheorem{definition}{Definition}[section] % definition numbers are dependent on theorem numbers
\theoremstyle{example}
\newtheorem{example}[definition]{Example} % same for example numbers
\theoremstyle{theorem}
\newtheorem{theorem}[definition]{Theorem}
\newtheorem{lemma}[definition]{Lemma}
%\newtheorem{defn}{definition}[section]
%\newtheorem{exmp}{example}[section]
%\newtheorem{rem}{remark}

\pgfplotsset{compat=1.11,
	/pgfplots/ybar legend/.style={
		/pgfplots/legend image code/.code={
			\draw[##1,/tikz/.cd,bar width=3pt,yshift=-0.2em,bar shift=0pt]
			plot coordinates {(0cm,0.8em)};
		},
	},
}

