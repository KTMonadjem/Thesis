\section{Discussion}
This chapter presents techniques to increase the safety of \ac{AV} systems. 
The first technique involves creating \ac{MNN} ensembles, consisting of multiple \acp{CNN}, which work together to increase the prediction accuracy.
The second technique uses runtime enforcement to verify the integrity of the \ac{MNN}'s classifications, attempting to detect misclassifications dynamically and put the vehicle in safe state if one is detected.
These two techniques work in tandem to greatly increase the safety of the \ac{AV} system where object misclassification is concerned.

Creating a \ac{MNN}  with three synchronous \acp{CNN}, forming a \ac{MNN} ensemble, did have a large effect on the classification accuracy of the system.
The classification accuracy was higher than that of an ordinary \ac{CNN} regardless of the number of epochs trained, but it was noted that the increase in accuracy was smaller the better trained the ordinary \ac{CNN} was.
This means that using \ac{MNN} ensembles is more effective for poorly or incompletely trained \acp{CNN}, but it still increases the prediction accuracy even when extensively trained \acp{CNN} are used.
However, implementing three \acp{CNN} is very resource intensive; each has to be trained separately to a suitable standard and on different data, each has to be implemented separately and each has to run separately.
This takes much more time, memory and processing power than implementing a single \ac{CNN} to classify the input images.
The \ac{MNN} used more than three times the resources that a single \ac{CNN} would use, but it can be argued that this trade-off is worth the increase in classification accuracy where human lives are concerned.

The second technique, the runtime enforcer, increased the prediction accuracy by 43\% for an extensively trained \ac{MNN} when input perturbations are involved.
The system only improved by 6\% when there were no input perturbations.
The enforcer provided an increase in performance regardless of the amount of training the \ac{MNN} had.
With no training, the classification accuracy of the system matched the accuracy of the \ac{LiDAR} sensor.
As the system trained, and the \ac{MNN} became more confident with its decisions, the accuracy of the system began to match the accuracy of the \ac{MNN}.
If the system were to be trained for more than 10,000 epochs, and the accuracy of the \ac{MNN} exceeded the accuracy of the \ac{LiDAR}, the overall accuracy of the system would likely increase beyond the accuracy of the \ac{LiDAR}.
More extensive training and testing of this system is proposed for future work.

%Using an enforcer to combine the \ac{MNN} output instead of another ANN
The biggest downside to this system is that it uses three different \acp{MNN} to run.
Each \ac{MNN} is an ensemble of \acp{CNN}, as introduced previously.
This means that using this layout increases the resource usage of the system by 300\%, as opposed using a single \ac{MNN} ensemble, or 900\% compared to using a single \ac{CNN}.
However, the trade-off for the safety of the system is well worth the extra resource cost.
Future work could look into decreasing the resource intensity of this system.

%\todo{Read memomcode paper on functional ANNs}
%\todo{Direct comparison to AI2}
%\todo{Train ANNs more and start at 1k epochs}

