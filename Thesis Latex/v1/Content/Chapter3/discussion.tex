\section{Discussion}
Deep \acfp{ANN}~\cite{schmidhuber2015deep}, \acp{ANN} with large, complex inputs (such as images) and a large number of layers are difficult to statically verify due to their complex nature~\cite{Gehr2018AI2SA}. 
Lots of work has been put into the verification of \acp{ANN}, but the results yielded from this work have many limitations and are generally time consuming.

Previously, this thesis introduced the concept of using runtime enforcement to dynamically enforce the safety properties of an \ac{ANN} as it runs. 
While this works very well for simple, feed forward \acp{ANN}, this does not extend to more complicated \acp{CNN}.
The output of a \ac{CNN} cannot be enforced as the input images are not recognisable by a simple algorithm.

This chapter presents techniques to increase the safety of \ac{AV} systems. 
We create an \acf{AV} case study that focuses on the object detection of the \ac{AV}. 
To verify the safety of the system, we use \ac{MNN} ensembles, consisting of multiple \acp{CNN}, which work together to increase the prediction accuracy and runtime verification to verify the integrity of the \ac{MNN}'s classifications, attempting to detect misclassifications dynamically and put the vehicle in safe state if one is detected.
These two techniques work in tandem to greatly increase the safety of the \ac{AV} system where object misclassification is concerned.

Creating a \ac{MNN}  with three synchronous \acp{CNN}, forming a \ac{MNN} ensemble, did have a large effect on the classification accuracy of the system.
The classification accuracy was higher than that of the average, ordinary \ac{CNN}.
Using \ac{MNN} ensembles is more effective for poorly or incompletely trained \acp{CNN}, but does not offer an increase in classification accuracy when extensively trained \acp{CNN} are used.
The ensembles do, however address the issue of worst case trained \acp{ANN}, by effectively classifying at the same accuracy as a best case trained \ac{ANN} would.
Implementing three \acp{CNN} is more resource intensive; each has to be trained separately to a suitable standard and on different data, each has to be implemented separately and each has to run separately.
This takes more time, memory and processing power than implementing a single \ac{CNN} to classify the input images.
The \ac{MNN} ensembles in this case study used three times the resources that a single \ac{CNN} would use, but it can be argued that this trade-off is worth the increase in classification accuracy where human lives are concerned, as the ensemble tend to the worst case scenario.
The ensemble function used for these tests was a very basic function that went little beyond simple averaging.
With a better ensemble function for these \ac{MNN} ensembles, a more significant increase in accuracy may be seen.

The runtime verifier increased the prediction accuracy by 43\% for an extensively trained \ac{MNN} when input perturbations are involved, but only improved by 6\% when there were no input perturbations.
Regardless of the amount of training the \ac{MNN} ensembles had, the run-time verifier was still able to detect a large number of misclassifications.
With no training, the classification accuracy of the system matched the accuracy of the \ac{LiDAR} sensor.
As the system trained, and the \ac{MNN} became more confident with its decisions, the accuracy of the system began to match the accuracy of the \ac{MNN}.

This system showed that the functional verification of \acp{ANN} for \acp{CPS} was possible using the reactive \ac{RV}. 
As opposed to the static verification methods such as~\cite{Gehr2018AI2SA}~\cite{reluplex}, this technique does not need to view the internals of the \acp{ANN}, but rather regards the \acp{ANN} as a black box.
This allows easier, scalable verification of \ac{ANN} systems.
Additionally, ensembles were demonstrated to be a method to cater for the worst case trained \ac{ANN} using synchronous compositions of \ac{SNN}.
With more work on an averaging function, these ensembles could increase prediction accuracy even for the best case trained \ac{ANN}.








