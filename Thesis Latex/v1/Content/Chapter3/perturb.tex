\section{Verification of Deep Artificial Neural Networks}
Deep \acfp{ANN}~\cite{schmidhuber2015deep}, \acp{ANN} with large, complex inputs and a large, complex layer structure and multiple layers, are hard to verify due to their complex nature~\cite{Gehr2018AI2SA}. 
Lots of work has been put into the verification of \acp{ANN}, but the results yielded from this work have many limitations and are generally time consuming.

Previously, this thesis introduced the concept of using runtime enforcement to dynamically ``verify'' an \ac{ANN} as it runs. 
While this works very well for simple, feed forward \acp{ANN}, this does not extend to more complicated \acp{CNN}.

\subsection{Runtime Enforcement of Deep Artificial Neural Networks}
Runtime Enforcement, as a solution to the verification of \acp{ANN} and introduced in the previous chapter, only works when the inputs to the \ac{ANN} are known.
This works very well with feed forward \acp{ANN}, as their inputs are, generally, defined and known.
Take the AI-BRO \ac{ANN} discussed in Section~\ref{sec:motivating-example}; this \ac{ANN} has a vector of defined inputs representing the objects ``seen'' by the vehicle.
Since dangerous situations are defined by both inputs and outputs, the outputs of this \ac{ANN} can be enforced because both the inputs and outputs are known at all times.

Complex \ac{ANN} inputs (such as images for \acp{CNN}) are unknown, and the purpose of the \acp{ANN} is to classify said inputs.
Since the inputs cannot be defined as something recognisable, the decision made by the \ac{ANN} cannot be classified as dangerous because a dangerous situation is defined by known inputs and outputs.
Take a \ac{CNN} that classifies pedestrians; we are using the \ac{CNN} because the software cannot recognise whether the input image shows a pedestrian or not.
If the pedestrian is misclassified as a tree, the system has no way to know this based off of the inputs to the \ac{CNN} alone.
Thus, behaviour of \acp{ANN} with complex inputs (most commonly classification \acp{ANN}) cannot be enforced using only the inputs to the \ac{ANN}.

The solution proposed is to use runtime enforcement as a means to safely combine the outputs from various sensors in the system and use these to enforce the outputs of the \ac{ANN}.


\subsection{Input Perturbation for Deep Artificial Neural Networks}
This chapter also tackles another problematic aspect of using \acp{CNN}: input perturbations.
A group showed that very slight modifications to the input image of a \acf{CNN}, such as the discolouration of a few pixels, could cause the image to be misclassified~\cite{Gehr2018AI2SA, ann-pert}.
A \ac{CNN} can train to high accuracy, e.g. 99.99\%, on the training set, but simple perturbations to the \ac{CNN}'s input can lead to drastically reduced accuracy once the \ac{CNN} has been deployed.
This thesis shows two methods of increasing the prediction accuracy of an \ac{CNN} affected by input perturbations: the first is the above-mentioned sensor fusion with runtime enforcement and the other is using a \acf{MNN} ensemble of different \acp{CNN}, each classifying a different aspect of the input image and congregating to make a decision as a whole.


\subsection{Sensor Fusion and Runtime Verification for Deep Artificial Neural Networks}
Sensor fusion is a highly researched topic for increasing the accuracy of object detection \acp{CNN}~\cite{SensorFusion2017}. 
Sensor fusion is the combination of two or more different sensor types to increase the overall sensor detection accuracy of the system.
Where \acfp{AV} are concerned safety is of utmost importance, making this the biggest area of research for sensor fusion using \acp{CNN}.

The proposed approach to safe deep \acp{ANN} combines runtime enforcement and sensor fusion to safely increase the detection accuracy of \acp{CNN} in \acp{AV}.
The sensors in question are a 360$\circ$ \ac{LiDAR} sensor and three front-facing cameras.
The sensor outputs are combined synchronously using a synchronous runtime enforcer, however the runtime enforcer does not enforce the outputs of the detection system.
Rather, the runtime enforcer verifies the integrity of the detected outputs, and in the case of a possible misclassification the system is put into a safe mode where control of the vehicle is returned to the driver.
This type of runtime enforcement has been termed as ``runtime verification''.
Using a \acf{SSNN} as the \ac{CNN} in the \ac{AV} system, the system is kept synchronous, time predictable and easy to formalise.

Additionally, to reduce the impact of perturbations to the \ac{CNN}'s inputs, the \ac{SSNN} used in the system is actually a \ac{MNN} composed of three \ac{MNN} ensembles, each with three synchronous \acp{CNN}.
This \ac{MNN} aims to greatly increase the prediction accuracy by splitting the prediction process amongst multiple, different \acp{CNN} such that the weakness of each \ac{CNN} is addressed by the other \acp{CNN}.














