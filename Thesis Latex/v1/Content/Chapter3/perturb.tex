\section{Verification of Deep Artificial Neural Networks}
Deep \acfp{ANN}~\cite{schmidhuber2015deep}, \acp{ANN} with large, complex inputs and a large, complex layer structure and multiple layers, are hard to verify due to their complex nature~\cite{Gehr2018AI2SA}. 

Need more here...

%\subsection{Verifying Artificial Neural Networks for Input Perturbations}

\subsection{Runtime Enforcement of Deep Artificial Neural Networks}
Runtime Enforcement, as a solution to the verification of \acp{ANN} and introduced in the previous chapter, only works when the inputs to the \ac{ANN} are known. 
Complex \ac{ANN} inputs (such as images) are unknown, the purpose of the \acp{ANN} is to classify said inputs.
Thus, behaviour of \acp{ANN} with complex inputs (most commonly classification \acp{ANN}) can not be enforced. 


\subsection{Input Perturbation for Deep Artificial Neural Networks}
These \acp{ANN} can train to high accuracy on the training set, but simple perturbations to the \ac{ANN}'s input can lead to drastically reduced accuracy.
A group showed that very slight modifications to an input image to a \acf{CNN}, such as discolouring a few pixels from the original image, could cause misclassification of the image~\cite{Gehr2018AI2SA}.

\subsection{Sensor Fusion and Runtime Verification for Deep Artificial Neural Networks}
Sensor fusion is a highly researched topic for increasing the accuracy of object detection \acp{CNN}~\cite{SensorFusion2017}. 
Sensor fusion is the combination of two or more different sensor types to increase the overall sensor detection accuracy of the system.
Where \acfp{AV} are concerned safety is of utmost importance, making this the biggest area of research for sensor fusion using \acp{CNN}.

The proposed approach to safe deep \acp{ANN} combines runtime enforcement and sensor fusion to safely increase the detection accuracy of \acp{CNN} in \acp{AV}.
The sensors in question are a 360$\circ$ \ac{LiDAR} sensor and three front-facing cameras.
The sensor outputs are combined synchronously using a synchronous runtime enforcer, however the runtime enforcer does not enforce the outputs of the detection system.
Rather, the runtime enforcer verifies the integrity of the detected outputs, and in the case of a possible misclassification the system is put into a safe mode where control of the vehicle is returned to the driver.
This type of runtime enforcement has been termed as ``runtime verification''.
Using \acfp{SSNN} as the \acp{CNN} in the \ac{AV} system, the system is kept synchronous, time predictable and easily formalised.














