\section{A Object Detection Case Study for Perturbed Inputs}
Research in \acfp{AV} systems is rapidly expanding, with companies such as Uber and Tesla the biggest contributors to this field.
With the growing use of \acp{AV} comes an increase in accidents related to these vehicles.
The majority of these accidents have one thing in common: a misclassification occurred right before the accident.
Input perturbations greatly decrease accuracy, thereby increasing the chance of a misclassification and, by proxy, the chance of an accident.

\subsection{Object Misclassification in Autonomous Vehicle Systems}
The inspiration for this case study was taken from the more recent Uber and Tesla \ac{AV} accident, such as~\cite{coldewey_2018}.
In these accidents, a misclassification, or series thereof, precedented the crash.
In the case of the Uber accident~\cite{coldewey_2018} a pedestrian crossing the road (at night) was misclassified a minimum of three times before the accident occurred.

Theoretically, this accident could have been prevented in two ways: the first being that no misclassifications occurred and the second being that the system recognised the misclassifications sooner and acted accordingly.
The misclassifications could have occurred due to adversarial perturbations of the input image due to the darkness. 
A system can be implemented to increase prediction accuracy regardless of the state of the inputs, perturbed or not.
By implementing a runtime verifier and a \acf{VDTA} (discussed in Chapter 4) to define the safety policy, misclassifications can be detected quickly and effectively.

\subsection{Runtime Verification of \acfp{CNN}}
Runtime Enforcement, as a solution to the verification of \acp{ANN} and introduced in the previous chapter, only works when the inputs to the \ac{ANN} are known.
This works very well with feed forward \acp{ANN}, as their inputs are, generally, defined and known.
Take the AI-BRO \ac{ANN} discussed in Section~\ref{sec:motivating-example}; this \ac{ANN} has a vector of defined inputs representing the objects ``seen'' by the vehicle.
Since dangerous situations are defined by both inputs and outputs, the outputs of this \ac{ANN} can be enforced because both the inputs and outputs are known at all times.

\ac{RE} can only function with well-defined inputs and outputs.
For instance, the \ac{ESS} system is well-defined, because we can trust current and voltage measurements, and we know that a given model of battery has a minimum state of charge.
We can enforce behaviour over a system relying on components such as these because constraints can be well defined.

However, systems that rely on ill-defined inputs, such as camera images, are not so straightforward.
How should a rule that says "stop the car if you detect a pedestrian" be implemented?
In the previous section, we used \ac{RE} to enforce a controller \ac{MLP} with well-defined inputs and output, but this isn't directly applicable in a real-world situation.
\acp{CNN} are used as image classifiers because we don't have a guaranteed method for doing this analysis.
An enforcer has no way of detecting if any one output from a \ac{CNN} is wrong, as it has no baseline of truth to operate from nor has any way of knowing what to change a value to in the event of a violation.

\acf{RV} provides the formal methodology for detecting faults that Runtime Enforcement relies upon.
By utilising this framework, rather than \ac{RE}, we can instead define, detect, and flag "anomalous" or "unsafe" situations.
For an \ac{AV} system, this would correspond to the system then returning control to the human driver.
The solution proposed is to use a run-time verifier to verify the outputs of a complex classifier \ac{MNN}.

\subsection{Sensor Fusion and Runtime Verification for an \acf{AV}~\cite{SensorFusion2017}}
Sensor fusion is a technique for increasing the accuracy of object detection \acp{CNN}. 
Sensor fusion is the combination of two or more different sensor types to increase the overall sensor detection accuracy of the system.
In \acfp{AV} safety is of utmost importance, and we consider the issue of input perturbations in such systems.

The proposed approach to safe deep \acp{ANN} combines \acf{RV}~\cite{runtime-verify} and sensor fusion to safely increase the detection accuracy of \acp{CNN} in \acp{AV}.
The sensors in question are a $360^\circ$ \ac{LiDAR} sensor and three front-facing cameras.
The sensor outputs are combined synchronously using a synchronous runtime verifier.
The runtime verifier checks the integrity of the detected outputs, and in the case of a possible misclassification the system is put into a safe mode where control of the vehicle is returned to the driver.
Using a \acf{SNN} as the \ac{CNN} in the \ac{AV} system, the system is kept synchronous, time predictable and easy to formalise.

Additionally, to reduce the impact of perturbations to the \ac{CNN}'s inputs, a \ac{MNN} composed of three \ac{MNN} ensembles is used, each with three synchronous \acp{CNN}.
This \ac{MNN} aims to greatly increase the prediction accuracy by splitting the prediction process amongst multiple, different \acp{CNN} such that the weakness of each \ac{CNN} is addressed by the other \acp{CNN}.

\subsection{An \acf{AV} Object Detection System}
Using the Tesla and Uber incidents as inspiration, we construct an environment to emulate a crash situation where adversarial inputs could cause a \ac{CNN}-controlled \ac{AV} to misclassify a roadside obstacle and crash.
As such, a simulation of an \ac{AV} was made to try and capture the intricacies of such a system where safety is concerned, while showing the efficacy of the proposed techniques.
Unlike the previous chapter, this \ac{AV} system focuses only on the object classification module of an \ac{AV} controller.
This \ac{AV} controller consisted of 3 \ac{MNN} ensembles, similar to the ensembles used in Chapter 4.
Each \ac{MNN} ensemble aims to classify a different aspect of the input image: the first classifies the colour of the object; the second classifies the shape of the object; and the last classifies the type of object.
These \acp{MNN} were trained and implemented using the \ac{VOC} 2012~\cite{pascal-voc-2012} and \ac{GTSRB}~\cite{Stallkamp2012-gtsrb} datasets.
Each ensemble aims to increase the overall classification accuracy and the run-time verifier provides an overall verdict of the object type being detected.
Using these three different classifications, a run-time verifier can more accurately give a verdict on the type of object classified by the third \ac{MNN} ensemble.

