\section{A Object Detection Case Study for Perturbed Inputs}
Research in \acfp{AV} systems is rapidly expanding, with companies such as Uber and Tesla the biggest contributors to this field.
With the growing use of \acp{AV} comes an increase in accidents related to these vehicles.
The majority of these accidents have one thing in common: a misclassification occurred right before the accident.
Input perturbations greatly decrease accuracy, thereby increasing the chance of a misclassification and, by proxy, the chance of an accident.

\subsection{Object Misclassification in Autonomous Vehicle Systems}
The inspiration for this case study was taken from the more recent Uber and Tesla \ac{AV} accident, such as~\cite{coldewey_2018}.
In these accidents, a misclassification, or series thereof, precedented the crash.
In the case of the Uber accident~\cite{coldewey_2018} a pedestrian crossing the road (at night) was misclassified a minimum of three times before the accident occurred.

Theoretically, this accident could have been prevented in two ways: the first being that no misclassifications occurred and the second being that the system recognised the misclassifications sooner and acted accordingly.
The misclassifications could have occurred due to adversarial perturbations of the input image due to the darkness. 
A system can be implemented to increase prediction accuracy regardless of the state of the inputs, perturbed or not.
By implementing a runtime verifier and a \acf{VDTA} (discussed in Chapter 4) to define the safety policy, misclassifications can be detected quickly and effectively.

\subsection{Runtime Verification of \acfp{CNN}}
Runtime Enforcement, as a solution to the verification of \acp{ANN} and introduced in the previous chapter, only works when the inputs to the \ac{ANN} are known.
This works very well with feed forward \acp{ANN}, as their inputs are, generally, defined and known.
Take the AI-BRO \ac{ANN} discussed in Section~\ref{sec:motivating-example}; this \ac{ANN} has a vector of defined inputs representing the objects ``seen'' by the vehicle.
Since dangerous situations are defined by both inputs and outputs, the outputs of this \ac{ANN} can be enforced because both the inputs and outputs are known at all times.

Images, as inputs to \acfp{CNN}, are unable to be recognised by a simple algorithm, the purpose of such \acp{CNN} is to classify image inputs.
Since the inputs cannot be defined as something recognisable, the decision made by the \ac{ANN} cannot be classified as dangerous because a dangerous situation is defined by known inputs and outputs.
Take a \ac{CNN} that classifies pedestrians; we are using the \ac{CNN} because the software cannot recognise whether the input image shows a pedestrian or not.
If the pedestrian is misclassified as a different object, the system has no way to know this based off of the inputs to the \ac{CNN} alone.
Thus, behaviour of \acp{CNN} cannot be enforced using only the inputs to the \ac{ANN}.

The solution proposed is to use runtime verification as a means to safely combine the outputs from various sensors in the system and use these to verify the outputs of the \ac{ANN}.

\subsection{Sensor Fusion and Runtime Verification for an \acf{AV}~\cite{SensorFusion2017}}
Sensor fusion is a technique for increasing the accuracy of object detection \acp{CNN}. 
Sensor fusion is the combination of two or more different sensor types to increase the overall sensor detection accuracy of the system.
In \acfp{AV} safety is of utmost importance, and we consider the issue of input perturbations in such systems.

The proposed approach to safe deep \acp{ANN} combines \acf{RV}~\cite{runtime-verify} and sensor fusion to safely increase the detection accuracy of \acp{CNN} in \acp{AV}.
The sensors in question are a $360^\circ$ \ac{LiDAR} sensor and three front-facing cameras.
The sensor outputs are combined synchronously using a synchronous runtime verifier.
The runtime verifier checks the integrity of the detected outputs, and in the case of a possible misclassification the system is put into a safe mode where control of the vehicle is returned to the driver.
Using a \acf{SNN} as the \ac{CNN} in the \ac{AV} system, the system is kept synchronous, time predictable and easy to formalise.

Additionally, to reduce the impact of perturbations to the \ac{CNN}'s inputs, a \ac{MNN} composed of three \ac{MNN} ensembles is used, each with three synchronous \acp{CNN}.
This \ac{MNN} aims to greatly increase the prediction accuracy by splitting the prediction process amongst multiple, different \acp{CNN} such that the weakness of each \ac{CNN} is addressed by the other \acp{CNN}.

\subsection{An \acf{AV} Object Detection System}
Replicating an \ac{AV} system, such as Uber's or Tesla's, would be very difficult in a thesis; it is expensive to purchase the technology such as \ac{LiDAR} and the machine vision cameras, and the time taken to implement the system as a single researcher is also not feasible.
As such, a simulation of an \ac{AV} was made to try and capture the intricacies of such a system where safety is concerned, while showing the efficacy of the proposed techniques.
Unlike the previous chapter, this \ac{AV} system focuses only on the object classification module of an \ac{AV} controller.
This \ac{AV} controller consisted of 3 \ac{MNN} ensembles, similar to the ensembles used in Chapter 4.
Each \ac{MNN} ensemble aims to classify a different aspect of the input image: the first classifies the colour of the object; the second classifies the shape of the object; and the last classifies the type of object.
These \acp{MNN} were trained and implemented using the \ac{VOC} 2012~\cite{pascal-voc-2012} and \ac{GTSRB}~\cite{Stallkamp2012-gtsrb} datasets.
Each ensemble aims to increase the overall classification accuracy and the run-time verifier provides an overall verdict of the object type being detected.
Using these three different classifications, a run-time verifier can more accurately give a verdict on the type of object classified by the third \ac{MNN} ensemble.

