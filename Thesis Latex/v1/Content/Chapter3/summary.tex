\section{Summary}
In the previous chapter of this thesis, runtime enforcement was introduced to dynamically increase the safety of \acfp{ANN}.
However, runtime enforcement is not always applicable to \ac{ANN} systems; most notably where \acp{ANN} with complex inputs (such as \acfp{CNN}) are concerned.
Thus, runtime enforcement is not a suitable solution to systems where \acp{CNN} are used, such as \acfp{AV}.

Inspiration for the safety of \acp{AV} was taken from the recent \ac{AV} accidents by Uber and Tesla vehicles.
These accidents resulted in the loss of life due to misclassifications made by the system controller, which consists of some layout of \acp{CNN} and other intelligent systems.
Additionally, perturbations to a \ac{CNN}'s input can greatly decrease the prediction accuracy of a \ac{CNN}.
This, in turn, increases the chance of a misclassification by the system's controller and decreases the safety of the system.

In this chapter, two techniques are implemented to increase the safety of \acfp{CNN} in \acf{AV} systems.
I.e. two techniques are introduced that make a system that misclassifies data safer: (1) reduce the chance of misclassifications and/or (2) act in a safe manner should a misclassification occur.

To address the first, a complex \acf{MNN} was implemented as the system's controller, see Figure~\ref{fig:mnn}.
This \ac{MNN} consisted of three different \ac{MNN} ensembles, each ensemble classifying a different aspect of the detected object.
Each \ac{MNN} ensemble consisted of three different synchronous \acp{CNN}, with the outputs of each \ac{CNN} combined to increase the prediction accuracy of the system.
The three ensembles' outputs were then combined synchronously during runtime to further increase the prediction accuracy of the system controller.
This \ac{MNN} also served to increase the prediction accuracy of the system when perturbations were affecting the inputs to the \ac{MNN}.

A system was created to simulate an \ac{AV} system as closely as was possible with the available resources.
The simulated system used the above-mentioned \ac{MNN} and sensor fusion to address the safety of the system, managed by a safety policy and enforced by a runtime enforcer.
The system combined \acf{LiDAR} input with the \ac{MNN}, thereby increasing the chance of detecting a misclassification.
Additionally, a timed automaton was used as the safety policy; this allowed the system to check for consecutive misclassifications and enforce the control of the vehicle where necessary.

The runtime enforcement used in this system hs been termed ``runtime verification'', as the outputs of the \ac{MNN} itself were not enforced.
Rather, the control of the vehicle (autonomous or driver operated) was enforced.
If a situation was deemed unsafe by the safety policy, the runtime enforcer would force control of the vehicle to the driver.

This system was tested and trained using a large combination of images from the \ac{VOC} and \ac{GTSRB} datasets, with images of cars and people taken from the \ac{VOC} dataset and images of various traffic signs taken from the \ac{GTSRB} dataset.
The system showed that input perturbations can reduce the accuracy of the system by as much as 45\% for a fully trained system.
The efficacy of the implemented techniques are shown: both with and without the enforcer, the accuracy of the system controller greatly improves, with the enforced policy catching 45 out of every 57 misclassifications.

























