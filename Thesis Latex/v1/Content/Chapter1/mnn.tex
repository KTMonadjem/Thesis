\section{Meta Neural Networks}
\label{sec:concurrent-sann}

%Compared with asychronous implementations, such as C's \texttt{pthread}, it is very simple to create concurrent `parallel' networks in a synchronous system. 
%This is because it is simple to express parallel \acp{ANN} as logically concurrent.
%This logical concurrency will be then compiled away to produce a
%single thread, although, as mentioned in
%Section~\ref{sec:motivating-example}, there are approaches which can
%create truly parallel implementations.

Complex \ac{CPS} use several interacting \acp{NN} to achieve their
functionality. However, to the best of our knowledge, formal modelling
and timing analysis of such concurrent \acp{NN} is lacking. To this
end we propose several alternative architectures, facilitated by the
synchronous approach for formalising the composition and timing of
several interacting \acp{NN}. We term these as \emph{meta neural networks}.
We start by formalising the input-output compatibility requirements.

\begin{definition}
	\label{def:io-compatibility}
	Let $\mathcal{M}$ be a set of \acp{SNN} according to Definition~\ref{def:bb-mlp}.
	We term $M_1 = \langle I_1, O_1, N_1, L_1, \alpha_1, \eta_1 \rangle$, $M_2=\langle I_2, O_2, N_2, L_2, \alpha_2, \eta_2  \rangle$ 
	$\in \mathcal{M}$ input-output compatible,
	IO-compatible for short, when the following hold:
	\begin{itemize}
		\item \emph{Size and type compatibility}: $|I_2|=|O_1|$ and the type of every connection also matches.
		\item \emph{Timing compatibility}: The outputs of $M_1$ are produced in a tick on or before $M_2$
		executes.
	\end{itemize}
\end{definition} 

Hence, just as we are able to break execution of \acp{ANN} over
multiple ticks by calling them 
``layer by layer'', we can also structure the synchronous program such
that it runs a set of concurrent \acp{SNN} ``network by network''.  We term such an arrangement of 
multiple \acp{SNN} as \emph{meta-layers} of \acp{SNN}.

\begin{definition}
	\label{def:nsanns}
	A network of \acp{SNN} organised into meta-layers is a tuple $H = \langle \mathcal{I}, \mathcal{O}, \mathcal{M}, \mathcal{L}, \mathcal{A}, \Delta \rangle$ where:
	\begin{itemize}
		\item We term the first meta-layer as the input
		meta-layer and the last one as the output
		meta-layer.
		\item  $\mathcal{I}=I_1 \cup .. \cup I_k$,
		where $I_1..I_k$ are the inputs of the \acp{SNN} in
		the input meta layer. If $K=|\mathcal{I}|$ then the
		domain is $\mathbf{I} = \mathbb{R}^K$
		\item $\mathcal{O}=O_1 \cup .. \cup O_j$,
		where $O_1..O_j$ are the outputs of the \acp{SNN} in
		the output meta layer. If $J=|\mathcal{O}|$ then the domain is $\mathbf{O} = \mathbb{R}^J$
		\item $\mathcal{M}$ is the set $M \in \mathcal{M}$ of
		\acp{SNN} according to Definition~\ref{def:bb-mlp}
		such any two connected \acp{SNN} are IO-compatible.
		\item $\mathcal{L}$ is the set of meta-layers of \acp{SNN}
		\item $\mathcal{A}: \mathcal{M} \rightarrow \mathcal{L}$ is the network mapping function that maps a given \ac{SNN} to a layer.
		\item $\Delta \subset \mathcal{M} \times \mathcal{M}$ provides the mapping of IO connectivity information between \acp{SNN} of different meta-layers. %@Partha: I am pretty sure we need a place to store the connections between SNNs - no?
	\end{itemize}
\end{definition}

\begin{figure}[H]
	\begin{subfigure}[t]{0.5\textwidth}
		\centering
		\scalebox{0.8}{\begin{tikzpicture}[->,>=stealth',shorten >=1pt,auto,
node distance=3cm,
semithick,scale=0.8, transform shape,scale=0.6, font=\LARGE]

\tikzstyle{every state}=[rectangle, minimum width=2cm, minimum height=1.5cm, text centered, 
fill=blue!20,draw=none,text=black, draw,line width=0.3mm, font=\LARGE]

%start state
\node[state]
(n1) 
{$M_1$}; 

\node[state]
(n2) [right of=n1]
{$M_2$};

\node[state]
(n3) [right of=n2]
{$M_3$};

\node[state,fill=none, draw=none]
(el) [right of=n3]
{...};

\node[state]
(nn) [right of=el]
{$M_n$};

\path[->] (n1) edge (n2);
\path[->] (n2) edge (n3);
\path[-] (n3) edge (el);
\path[->] (el) edge (nn);

\end{tikzpicture}
}
		\caption{Pipeline: $WCRT = M_1 \oplus M_2 \oplus M_3 \oplus \ldots \oplus M_n$}
		\label{fig:tca-nn-pipeline}
	\end{subfigure}
	
	\vspace{3mm}
	\begin{subfigure}[t]{0.5\textwidth}
		\centering
		\scalebox{0.8}{\begin{tikzpicture}[->,>=stealth',shorten >=1pt,auto,
node distance=3cm,
semithick,scale=0.8, transform shape,scale=0.6, font=\LARGE]

\tikzstyle{every state}=[rectangle, minimum width=2cm, minimum height=1.5cm, text centered, 
fill=blue!20,draw=none,text=black, draw,line width=0.3mm, font=\LARGE]

\tikzstyle{every ellips}=[fill=none,text=black,line=none,font=\LARGE]

%start state
\node[state]
(n1)
{$M_1$}; 

\node[state]
(n2) [below of=n1]
{$M_2$};

\node[state]
(n3) [right of=n1]
{$M_3$};

\node[state]
(n4) [right of=n2]
{$M_4$}; 

\node[state]
(n5) [right of=n3]
{$M_5$};

\node[state]
(n6) [right of=n4]
{$M_6$}; 

\node[state,fill=none, draw=none]
(el1) [right of=n5]
{...};

\node[state,fill=none, draw=none]
(el2) [right of=n6]
{...};

\node[state]
(nn) [right of=el1]
{$M_n$};

\node[state]
(nn1) [right of=el2]
{$M_{n+1}$};

\path[->] (n1) edge (n3);
\path[->] (n1) edge (n4);
\path[->] (n2) edge (n3);
\path[->] (n2) edge (n4);
\path[->] (n3) edge (n5);
\path[->] (n3) edge (n6);
\path[->] (n4) edge (n5);
\path[->] (n4) edge (n6);
\path[-] (n5) edge (el1);
\path[-] (n5) edge (el2);
\path[-] (n6) edge (el1);
\path[-] (n6) edge (el2);
\path[->] (el1) edge (nn);
\path[->] (el1) edge (nn1);
\path[->] (el2) edge (nn);
\path[->] (el2) edge (nn1);


\end{tikzpicture}
}
		\caption{Grid: $WCRT = \left(M_1 \odot M_2\right) \oplus \left(M_3 \odot M_4\right) \oplus \ldots \oplus \left(M_n \odot M_{n+1}\right)$}
		\label{fig:tca-nn-grid}
	\end{subfigure}
	
	\vspace{3mm}
	\begin{subfigure}[t]{0.5\textwidth}
		\centering
		\scalebox{0.8}{\begin{tikzpicture}[->,>=stealth',shorten >=1pt,auto,
node distance=3cm,
semithick,scale=0.8, transform shape,scale=0.6, font=\LARGE]

\tikzstyle{every state}=[rectangle, minimum width=2cm, minimum height=1.5cm, text centered, 
fill=blue!20,draw=none,text=black, draw,line width=0.3mm, font=\LARGE]

\tikzstyle{every ellips}=[fill=none,text=black,line=none,font=\LARGE]

%start state
\node[state]
(n1)
{$M_1$}; 

\node[state]
(n2) [below of=n1]
{$M_2$};

\node[state]
(n3) [right of=n1, yshift=-1.5cm]
{$M_3$};

\node[state]
(n4) [right of=n3, yshift=1.5cm]
{$M_4$};

\node[state]
(n5) [below of=n4]
{$M_5$};

\node[state]
(n6) [right of=n4, yshift=-1.5cm]
{$M_6$};

\node[state,draw=none,fill=none]
(el1) [right of=n6, yshift=1.5cm]
{...}; 

\node[state,draw=none,fill=none]
(el2) [below of=el1]
{...}; 


\path[->] (n1) edge (n3);
\path[->] (n2) edge (n3);
\path[->] (n3) edge (n4);
\path[->] (n3) edge (n5);
\path[->] (n4) edge (n6);
\path[->] (n5) edge (n6);
\path[->] (n6) edge (el1);
\path[->] (n6) edge (el2);

\end{tikzpicture}
}
		\caption{2x1: $WCRT = \left(M_1 \odot M_2\right) \oplus M_3 \oplus \left(M_4 \odot M_5\right) \oplus M_6 \oplus \ldots$}
		\label{fig:tca-nn-someparallel}
	\end{subfigure}
	
	\caption{Different concurrent \ac{ANN} arrangements}
	\label{fig:tca-nn}
\end{figure}

Depending on the application several options may be explored when
considering the design space for the concurrent arrangement of
multiple networks. Some of these are presented in Figure~\ref{fig:tca-nn}.
The first approach, in Figure~\ref{fig:tca-nn-pipeline}, creates a pipeline of networks, where they are simply executed in sequence.
Alternatively, \acp{SNN} could be arranged in parallel, as in Figure~\ref{fig:tca-nn-grid}. 
Finally, an arrangement might be more arbitrary, as in Figure~\ref{fig:tca-nn-someparallel}. 
Due to the synchronous approach, as long as the \emph{size and type
	compatibility} of Definition~\ref{def:io-compatibility} is
satisfied, it is simple to achieve \emph{timing compatibility} based on Definition~\ref{def:timing-compatibility}. 
Moreover, the worst case timing behaviour of each composition is
straight forward to represent as a formula in \ac{WCRT}-algebra, as
shown in Figure~\ref{fig:tca-nn}.

\begin{definition}
	\label{def:timing-compatibility}
	Timing compatibility between \acp{SNN} in Figure~\ref{fig:tca-nn} may
	be achieved using the following:
	\begin{itemize}
		\item When a \ac{SNN} $M_1$'s output is connected to $M_2$ and the
		period of $M_1$ is $n$-ticks, then for timing compatibility, $M_2$
		has $n$-\texttt{pause} statements at the start.
		\item When a \ac{SNN} $M_1$ and $M_2$'s output is connected to $M_3$ and the
		period of $M_1$ is $n_1$-ticks and that of $M_2$ is $n_2$-ticks,
		then $M_3$ has $(n_1 \oplus n_2)$-\texttt{pause} statements at the start.
		\item When a connectivity self-loop is provided in $\Delta$, we break it explicitly.
	\end{itemize}
\end{definition}

%Contrast this with an asynchronous approach, which could require extensive usage of mutexes and locking mechanisms!

%\subsection{Interfaces of concurrent \acp{SNN}}

\ignore{
	When considering the interconnection of different \ac{SNN} in a network of \ac{SNN} $H$, the connections can only be considered valid iff the source and destination types match, and if the source of an interface is located `earlier' (i.e. on an earlier layer $l \in \mathcal{L}$ or in the global inputs $I$) than the destination interface. 
	More formally, this can be expressed as Definition~\ref{def:sann-conns}.
	
	\begin{definition}
		\label{def:sann-conns}
		A network of \acp{SNN} $H = \langle I, O, \mathcal{M}, \mathcal{L}, \mathcal{A}, \Delta \rangle$ is correctly interconnected iff $\forall \delta \in \Delta, \forall i_d \in I \in n_d \in \delta, \exists i_d \in \left(\mathcal{I} \cup \left(M_s \in \mathcal{M}|\mathcal{A}\left(M_s\right) < \mathcal{A}\left(M_d\right)\right)\right)$, where $i_d$ is the source of connection $\delta$, $M_s$ is the possible source network for the connection, and $M_d$ is the destination network.
		% this should read as:
		% for all connections in H, for all sources of those connections, the source exists in either (an output on a network in an earlier layer that the destination of the connection) or (the inputs I) 
	\end{definition}
}

\begin{figure}[H]
	\centering
	\scalebox{0.8}{\begin{tikzpicture}[->,>=stealth',shorten >=1pt,auto,
node distance=3cm,
semithick,scale=0.8, transform shape,scale=0.6, font=\huge]

\tikzstyle{every state}=[rectangle, minimum width=2cm, minimum height=1.5cm, text centered, 
fill=blue!20,draw=none,text=black, draw,line width=0.3mm, font=\LARGE]

\tikzstyle{every ellips}=[fill=none,text=black,line=none,font=\LARGE]

%start state
\node[state,fill=none]
(x1)
{$x_1$}; 

\node[state,fill=none]
(x2) [below of=x1]
{$x_2$}; 

\node[state,fill=none]
(x3) [below of=x2]
{$x_3$}; 

\node[state,fill=none]
(x4) [below of=x3]
{$x_4$}; 

\node[state]
(n1) [right of=x1, yshift=-1.5cm]
{$M_1$}; 

\node[state,draw=none,fill=none]
(inv) [below of=n1]
{}; 

\node[state]
(n2) [below of=inv]
{$M_2$};

\node[state]
(n3) [right of=inv]
{$M_3$};

\node[state,fill=none]
(y) [right of=n3]
{$y$};

\path[->] (x1) edge (n1);
\path[->] (x2) edge (n1);
\path[->] (x3) edge (n2);
\path[->] (x4) edge (n2);
\path[->] (n1) edge (n3);
\path[->] (n2) edge (n3);
\path[->] (n3) edge (y);

\end{tikzpicture}
}
	\caption{Simple three-network XOR system}
	\label{fig:three-xor}
\end{figure}

\begin{example}
	\label{ex:simple-xor}
	Consider three XOR \ac{MLP} networks $\mathcal{M} = \left(M_1, M_2, M_3\right)$ in a system $H$.
	They are arranged in the formation shown in Figure~\ref{fig:three-xor}, such that they compute $y = \big(x_1$ \textit{xor} $x_2\big)$ \textit{xor} $\big(x_3$ \textit{xor} $x_4\big)$.
	The overall network $H$ has two meta-layers in $\mathcal{L}$, an input layer and an output layer. 
	Hence, the global inputs $I = \langle x_1, x_2, x_3, x_4 \rangle$, and global outputs $O = \langle y \rangle$.
	The connectivity of $\Delta \in H$ is correctly defined, as the internal connections between $M_1$, $M_2$, and $M_3$ are matching, i.e. they are the same widths and types. 
	
	Let us execute the networks $M \in \mathcal{M}$ using the multi-periodic ``layer by layer'' approach. 
	Hence, for each network $M$, $WCRT = l_i \oplus l_h \oplus l_o$.
	Using \ac{WCRT} algebra, we can compute the system WCRT is of the form $WCRT = \left(M_1 \odot M_2\right) \oplus M_3$, ergo, the actual system \\ $WCRT = \Big(\left(l_1^i \oplus l_1^h \oplus l_1^o\right) \odot \left(l_2^i \oplus l_2^h \oplus l_2^o\right)\Big) \oplus \left(l_3^i \oplus l_3^h \oplus l_3^o\right)$ , where $l^x_y$ indicates layer $l_x \in L_y$ (in network $M_y$).
\end{example}

\ignore{
	Secondly, issues can be caused when considering \emph{alignment} of interconnected networks.
	If two parallel networks take differing numbers of cycles to execute, how can the outputs match cleanly to the inputs of subsequent layers?
	However, this proves to not be a difficult issue to resolve, as tick
	delays can be introduced to parallel networks $n \in l \in
	\mathcal{L}$ such that the number of cycles to execute each network,
	and thus to pass each data variable, is the same.}

The Example in~\ref{ex:timing} discusses the issue of timing compatibility.

\begin{figure}[H]
	\centering
	\scalebox{0.8}{\begin{tikzpicture}[->,>=stealth',shorten >=1pt,auto,
node distance=3cm,
semithick,scale=0.8, transform shape,scale=0.6, font=\LARGE]

\tikzstyle{every state}=[rectangle, minimum width=2cm, minimum height=1.5cm, text centered, 
fill=blue!20,draw=none,text=black, draw,line width=0.3mm, font=\LARGE]

\tikzstyle{every ellips}=[fill=none,text=black,line=none,font=\LARGE]

%start state
\node[state,fill=none]
(x1)
{$x_1$}; 

\node[state,fill=none]
(x2) [below of=x1]
{$x_2$}; 

\node[state,fill=none]
(x3) [below of=x2]
{$x_3$}; 

\node[state]
(n1) [right of=x1, yshift=-1.5cm]
{$M_1$}; 

\node[state]
(n2) [right of=n1, yshift=-1.5cm]
{$M_2$};

\node[state,fill=none]
(y) [right of=n2]
{$y$};

\path[->] (x1) edge (n1);
\path[->] (x2) edge (n1);
\path[->] (x3) edge (n2);
\path[->] (n1) edge (n2);
\path[->] (n2) edge (y);

\end{tikzpicture}
}
	\caption{Two-network XOR system}
	\label{fig:two-xor}
\end{figure}

\begin{example}
	\label{ex:timing}
	Consider the case of a two-input XOR system $H$, as presented in Figure~\ref{fig:two-xor}.
	This will compute $y = \big(x_1$ \textit{xor} $x_2\big)$ \textit{xor} $x_3$.
	As in Example~\ref{ex:simple-xor}, there are two meta-layers in $\mathcal{L}$. 
	Once again, the connectivity of $\Delta$ is correctly defined, as the inputs $I_1 \in M_1$ are simply in the global inputs $I$, and the inputs $I_2 \in M_2$ are provided as both a global input and as the output $O_1 \in M_1$.  
	
	However, as can be seen, the sources for $M_2$ will take different numbers of logical ticks to arrive.
	As a result, we must delay the input $x_3$ by some number of cycles.
	If we are using the multi-periodic ``layer by layer'' execution timing, this number will be $|L \in M_1|$, i.e. the number of ticks it will take to execute the layers of $M_1$ (in our case, $|L| = 3$ cycles). 
	Or, if we are using a mono-periodic black-box approach, then we will need to delay the input $x_3$ by one cycle.
\end{example}

All architectures of meta neural networks discussed thus far are
devoid of instantaneous cycles. While such cycles in asynchronous settings are difficult
to deal with, the synchronous approach offers two different
alternatives. Synchronous compilers can detect such static cycles and flag
them as \emph{non-causal}~\cite{benveniste2003synchronous} programs, and reject them outright. Alternatively, such instantaneous cycles
can be broken by inserting an unit delay (i.e. a \texttt{pause}
statement) to break the cycle.
%\todo{Partha: I don't like the usage of the word "cycle" in here, because we are using mono-cycles and multi-cycles already. Can we use something else?}

%\pr{Hammond -- please read this section and carefully examine the
%  Definitions. Please check that the terminology of your examples and
%  Figures are consistent with my definitions. One question I have
%  about this section is how these are related to the next section.}


%\todo{Discuss how this differs from Asynchronous arragement of ANNs}.
%\todo{Discuss IO with environment. Capture/compute/release, when to capture / release?}

%\todo{Discuss IO compatiblity. Problems: mismatched IO widths, types. Concurrent networks taking different numbers of cycles (sustaining data?)}.


