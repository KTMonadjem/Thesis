\section{\acf{MNN2C}}
\label{sec:mnn2c}
\acfp{MBD} is a traditional approach to designing systems~\cite{dmd2019}.
As opposed to design by trial-and-error, \ac{MBD} builds formally defined, safer systems~\cite{dmd2019}.
This approach to creating systems is highly preferred where the safety of the system, e.g. systems that interact with humans, is concerned.
This is relevance where \ac{AI} systems are created that interact with the same environment as humans, such as the \acf{ESS} shown in Section~\ref{sec:ess}.

The ability to generate formally defined system models from existing \acp{ANN} allows for potentially unsafe and unpredictable \acp{ANN} to be re-implemented in such a way that they are safe and predictable.
This allows the use of such \acp{ANN} in systems were safety is critical, e.g. \acfp{CPS}.

Keras~\cite{chollet2015keras} is an \ac{ANN} Python library, able to use TensorFlow as a backend. 
Keras enables the quick, easy and highly adaptable training of various \acp{ANN}.
The \acp{ANN} generated by Keras are not safe: they are not formally defined, and any timing must be done using measurement based timing. 
This poses a problem to the use of the \acp{ANN} in \ac{CPS}, since \ac{CPS} have strict rules and protocols that the software must follow.

\acf{MNN2C} is an \ac{ANN} compiler that aims to produce \acf{MNN} models in C using pre-existing, Keras-trained \acp{ANN}.
The C code that \ac{MNN2C} produces is not only time-predictable, but the generated \acp{MNN} are also formally defined in \ref{def:snn}.
This means that the \ac{MNN} models generated can be used in \ac{CPS} with the knowledge that these \acp{MNN} are rigorously, mathematically defined. 
Additionally, \ac{MNN2C} allows for the simple, easy and quick training of \acp{ANN} which can then be implemented in existing C systems.

\subsection{Future Work}
\ac{MNN2C} currently only generates time predictable, formally defined C code \acp{ANN} based off of Keras \acp{ANN}.
While this is a huge step, this is not the limit when it comes to using \acp{ANN}.
\ac{MNN2C} can be expanded to generate code that is implementable on various hardware platforms, such as \acp{FPGA}, graphics cards and many others.
This would allow for the implementation of \acp{ANN} in various \ac{CPS} besides those that just use C software.













