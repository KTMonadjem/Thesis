\section{Multi-Periodic \acp{SNN} in Esterel}
\label{sec:esterel-mapping}

The mono-periodic implementation of \ac{SNN} using the methodology in Listing~\ref{lst:ai-bro}, while being time predictable, has limitations.
This is because each \ac{NN} must run to completion before any reaction can complete. 
For large networks with many hidden layers, this is not ideal --- as
the overall \ac{WCRT} of the system could become quite large, causing
violations when considering response to some critical inputs which may
need immediate attention to ensure safety. We highlight this aspect
using an \acf{ESS} case study in Section~\ref{sec:results}.

There are other approaches for arranging the internals of each network while still meeting Definition~\ref{def:sann}, and to illustrate this, we will use a simple XOR (exclusive or) \ac{MLP}.
This network is depicted in Figure~\ref{fig:xor-ann}, and as can be seen, it features five neurons in three layers. While we use this simple illustration, the 
methodology is applicable for any \ac{NN}.

\begin{figure}[H]
	\centering
	\scalebox{0.8}{\def\layersep{2.25cm}
\begin{tikzpicture}[shorten >=1pt,->,draw=black!100, node distance=\layersep]
	\tikzstyle{every pin edge}=[<-,shorten <=1pt]
	\tikzstyle{neuron}=[circle,fill=black!25,minimum size=20pt,inner sep=0pt]
	\tikzstyle{input neuron}=[neuron, fill=white!100,draw=black];
	\tikzstyle{output neuron}=[neuron, fill=white!100,draw=black];
	\tikzstyle{hidden neuron}=[neuron, fill=white!100,draw=black];
	\tikzstyle{annot} = [text width=4em, text centered]
	
	% Draw the input layer nodes
	\foreach \name / \y in {1,...,2}
	% This is the same as writing \foreach \name / \y in {1/1,2/2,3/3,4/4}
	\node[input neuron, pin=left:Input $x_\y$] (I-\name) at (0,-\y) {$i_\y$};
	
	% Draw the hidden layer nodes
	\foreach \name / \y in {1,...,3}
	\path[yshift=0.5cm]
	node[hidden neuron] (H-\name) at (\layersep,-\y cm) {$h_\y$};
	
	% Draw the output layer node
	\node[output neuron,pin={[pin edge={->}]right:Output $y$}, right of=H-2] (O1) {$o_1$};
	
	% Connect every node in the input layer with every node in the
	% hidden layer.
	%\foreach \source in {1,...,2}
	%\foreach \dest in {1,...,3}
	%\path (I-\source) edge node[above] {1} (H-\dest);
	\path (I-1) edge node[above] {1} (H-1);
	\path (I-1) edge node[above] {0.5} (H-2);
	\path (I-2) edge node[above] {0.5} (H-2);
	\path (I-2) edge node[above] {1} (H-3);
	
	% Connect every node in the hidden layer with the output layer
	%\foreach \source in {1,...,3}
	\path (H-1) edge node[above] {1} (O1);
	\path (H-2) edge node[above] {-2} (O1);
	\path (H-3) edge node[above] {1} (O1);
	
	% Annotate the layers
	\node[annot,above of=H-1, node distance=2cm] (hl) {\textit{Hidden Layer}\\$l_h$};
	\node[annot,left of=hl] {\textit{Input Layer}\\$l_i$};
	\node[annot,right of=hl] {\textit{Output Layer}\\$l_o$};
\end{tikzpicture}}
	\caption{Example XOR \ac{MLP} \ac{ANN}.	\label{fig:xor-ann}}
\end{figure}

Each neuron in this network is based on the model in Figure~\ref{fig:artificial-neuron}, with a simple Boolean step activation function and zero bias. 
The activation function is represented by Equation~\ref{eqn:xor-neuron}, where $y$ is the output, $x \in X$ are the inputs, and $w \in W$ the weights.

\begin{equation}
y =
\begin{cases}
1 & \text{$\Sigma_{i} \left(x_i \times w_i\right) >= 1$} \\
0 & \text{otherwise}
\end{cases}
\label{eqn:xor-neuron}
\end{equation}

\begin{definition}
	\label{def:bb-mlp}
	An \ac{MLP} is formalised as a tuple $M = \langle I, O, N, L,
	\alpha, \eta  \rangle$, where:
	\begin{itemize}
		\item $I$ is a finite collection of input variables with
		its domain being $\mathbf{I} = \mathbb{R}^n$
		\item  $O$ is a finite collection of  output variables with
		its domain being $\mathbf{O} = \mathbb{R}^m$
		\item  $N$ denotes a set of neurons
		\item $L$ denotes a set of layers
		\item $\alpha: N \rightarrow L$ is the neuron mapping function
		that maps a given neuron to a layer
		\item $\eta: \mathbf{I} \rightarrow \mathbf{O}$ is the non-linear
		function (termed the network function) that provides the behaviour of a given network i.e.
		when provided a vector of input of size $n$ produces an
		vector of output of size $m$. Internally, 
	\end{itemize}
	\ignore{
		$i \in I$ is each input to the network, $o \in O$ is each output from the network, 
		$\eta$ is the set of neurons, $L$ is the set of layer functions $l:i \rightarrow o$ which perform intermediate data transforms, $m: \eta \rightarrow L$ is the neuron mapping function that associates a neuron to a given layer,
		and $f: I \rightarrow O$ is a function that captures the non-linear behaviour of the entire network 
		--- i.e. a ``black-box'' untimed non-linear transformation function which converts inputs to outputs via the execution of each layer of neurons in sequence.}
	%$m \subset \eta \times L$ provides the mapping of neurons to layers. 
	%	We can describe the operation of $n$ as $n\left(I\right) = O$, i.e. $n$ describes a `black-box' untimed non-linear transformation function which converts inputs to outputs via the execution of each layer of neurons.
\end{definition} 

\begin{example}
	The \ac{MLP} presented in Figure~\ref{fig:xor-ann} can be presented formally as $I=\langle x_1, x_2 \rangle$, $O = \langle y \rangle$, $N= \{i_1, i_2, h_1, h_2, h_3, o_1\}$ and $L
	= \{l_i, l_h, l_o\}$. The function $\alpha$ maps neurons to
	layers e.g. $\alpha(h_1) = l_1$ and the network function $\eta$
	given input $\langle 0, 1 \rangle$ produces an output $1$
	i.e. $\eta(\langle 0, 1 \rangle) = 1$.
	
\end{example}
%\pr{Illustrate this definition using Figure~\ref{fig:xor-ann}. Also, where is this definition used formally?}

Using Definition~\ref{def:bb-mlp}, we can implement the XOR \ac{ANN} as a ``black-box'' function in Esterel, as presented in Figure~\ref{fig:tca-bb-n}. 
The corresponding mono-periodic \ac{TCA} is shown beside the code.
The \ac{WCRT} is the time taken to execute the network function
$\eta$, i.e. $WCRT\left(M\right) = WCRT\left( \eta\right)$. 
In the following, we use $WCRT(\eta)$ and $\eta$ interchangeably.

\begin{figure}[H]
	\centering
	\begin{subfigure}[]{0.15\textwidth}
		\centering
		\begin{tikzpicture}[->,>=stealth',shorten >=1pt,auto,
node distance=2.5cm,
semithick,scale=0.8, transform shape,scale=0.6]

\tikzstyle{every state}=[ellipse, minimum width=2.5cm, minimum height=1.5cm, text centered, 
fill=blue!20,draw=none,text=black, draw,line width=0.3mm, font=\LARGE]

%start state
\node[state]
(init) 
{$\eta$}; 

\draw[->] (init.south) 
|- ($(init.south) + (0, -0.75cm)$) 
-| ($(init.east) + (1cm, 0)$) 
|- ($(init.east) + (0, 1.5cm)$) 
-| (init.north);

\end{tikzpicture}

	\end{subfigure}%
	\begin{subfigure}[]{0.33\textwidth}
		\vspace{3mm}
		\begin{lstlisting}
		loop
		emit y(run_XOR_ANN_n(?x1, ?x2));
		pause;
		end
		\end{lstlisting}
	\end{subfigure}
	\caption{Mono-periodic `black-box', $WCRT = \eta$}
	\label{fig:tca-bb-n}
\end{figure}

It can be noted that by Proposition~\ref{lemma1}, any \ac{ANN} implemented using the approach in Figure~\ref{fig:tca-bb-n} is also an instance of \ac{SNN} defined using Definition~\ref{def:sann}.
%\pr{This is same as the lemma in section 2. Do we need this?}

\begin{theorem}
	\label{thm:bb-soundness}
	(Soundness) A mono-periodic \ac{SNN} given an input vector $\mathbf{i}$
	produces an output vector $\mathbf{o}$ in a given tick \miff the corresponding \ac{MLP} 
	given the same input vector $\mathbf{i}$ produces the identical output vector $\mathbf{o}$ during any execution.
	
	%The synchronous execution of a ``black-box'' function $n$ will give the same output values $O$ for the same input values $I$ as the non-synchronous execution of the
	%\ac{MLP} $n$.
\end{theorem}

\begin{proof}
	The proof of Theorem~\ref{thm:bb-soundness} trivially follows from the
	fact that synchronous execution of a black-box function does not change
	the mathematical properties of that function. %, as all Esterel function
	%calls are state-less and  side effect free.
\end{proof}

\ignore{
	
	\begin{equation}
	y = f\Big(b + \sum_{i=1}^{n} \left(x_i \times w_i\right)\Big)
	\label{eqn:artificial-neuron}
	\end{equation}
	
	In the XOR example, each neuron has a \textit{threshold step function} as its activation function $f$, and has zero bias $b$.
	Hence, each neuron in the \textit{hidden} and \textit{output} layers is following Equation~\ref{eqn:xor-neuron}.
	
	\begin{example}
		In our XOR \ac{ANN}, let $i_1 = 1$ and $i_2 = 0$. \\
		Then, as $1 \times 1 \geqslant 1$, $h_1y = 1$. \\
		However, $\big(\left(1 \times 0.5\right) + \left(0 \times 0.5\right)\big) \ngeqslant 1$, so $h_2y = 0$.
		Likewise, $h_3y = 0$.  \\
		Then, as $\big((1 \times 1) + (0 \times -2) + (1 \times 1)\big) \ngeqslant 1$, the final output $o_1y = 1$.  \\
		This is correct, as $1$ \textit{xor} $0 = 1$.
	\end{example}
}

\begin{figure}
\begin{subfigure}[]{0.5\textwidth}
	\begin{subfigure}[]{0.3\textwidth}
		\centering
		\begin{tikzpicture}[->,>=stealth',shorten >=1pt,auto,
node distance=2.5cm,
semithick,scale=0.8, transform shape,scale=0.6, font=\huge]

\tikzstyle{every state}=[ellipse, minimum width=2cm, minimum height=1.5cm, text centered, 
fill=blue!20,draw=none,text=black, draw,line width=0.3mm, font=\LARGE]

%start state
\node[state]
(init) 
{$l_i \odot l_h \odot l_o$}; 

\draw[->] (init.south) 
|- ($(init.south) + (0, -0.75cm)$) 
-| ($(init.east) + (0.25cm, 0)$) 
|- ($(init.east) + (0, 1.5cm)$) 
-| (init.north);

\end{tikzpicture}

	\end{subfigure}%
	\begin{subfigure}[]{0.66\textwidth}
		\vspace{3mm}
		\begin{lstlisting}
loop
	call run_XOR_ANN_li()(?x1, ?x2);
	call run_XOR_ANN_lh();
	emit y(run_XOR_ANN_lo());
	pause;
end
		\end{lstlisting}
	\end{subfigure}
	\caption{Mono-periodic `layer by layer', $WCRT = l_i \odot l_h \odot l_o$}
	\label{fig:tca-bb}
\end{subfigure}

\vspace{5mm}
\begin{subfigure}[]{0.5\textwidth}
	\begin{subfigure}[]{0.3\textwidth}
		\centering
		\begin{tikzpicture}[->,>=stealth',shorten >=1pt,auto,
node distance=2.5cm,
semithick,scale=0.8, transform shape,scale=0.6, font=\huge]

\tikzstyle{every state}=[ellipse, minimum width=2.5cm, minimum height=1.5cm, text centered, 
fill=blue!20,draw=none,text=black, draw,line width=0.3mm, font=\LARGE]

%start state
\node[state]
(l0) 
{$l_i$};

\node[state]
(l1) [below of=l0]
{$l_h$}; 

\node[state]
(l2) [below of=l1]
{$l_o$};  

\path[->] (l0) edge (l1);
\path[->] (l1) edge (l2);

\draw[->] (l2.south) 
|- ($(l2.south) + (0, -0.75cm)$) 
-| ($(l2.east) + (1cm, 0)$) 
|- ($(init.east) + (0, 1.5cm)$) 
-| (init.north);

\end{tikzpicture}

	\end{subfigure}%
	\begin{subfigure}[]{0.66\textwidth}
	\begin{lstlisting}
loop
	call run_XOR_ANN_li()(?x1, ?x2);
	pause;
	call run_XOR_ANN_lh();
	pause;
	emit y(run_XOR_ANN_lo());
	pause;
end
	\end{lstlisting}
	\end{subfigure}
	\caption{Multi-periodic `layer by layer', $WCRT = l_i \oplus l_h \oplus l_o$}
	\label{fig:tca-layers}
\end{subfigure}

\vspace{5mm}
\begin{subfigure}[]{0.5\textwidth}
	\begin{subfigure}[]{0.3\textwidth}
		\centering
		\begin{tikzpicture}[->,>=stealth',shorten >=1pt,auto,
node distance=2.5cm,
semithick,scale=0.8, transform shape,scale=0.6, font=\huge]

\tikzstyle{every state}=[ellipse, minimum width=2.5cm, minimum height=1.5cm, text centered, 
fill=blue!20,draw=none,text=black, draw,line width=0.3mm, font=\LARGE]

%start state
\node[state]
(l0) 
{$i_1 \odot i_2$};

\node[state]
(l1) [below of=l0]
{$h_1 \odot h_2 \odot h_3$}; 

\node[state]
(l2) [below of=l1]
{$o_1$};  

\path[->] (l0) edge (l1);
\path[->] (l1) edge (l2);

\draw[->] (l2.south) 
|- ($(l2.south) + (0, -0.75cm)$) 
-| ($(l2.east) + (1.5cm, 0)$) 
|- ($(init.east) + (0, 1.5cm)$) 
-| (init.north);

\end{tikzpicture}

	\end{subfigure}%
	\begin{subfigure}[]{0.66\textwidth}
		\begin{lstlisting}
loop
	[
		call run_XOR_ANN_i1()(?x1);
		|| 
		call run_XOR_ANN_i2()(?x2);
	];
	pause;
	[
		call run_XOR_ANN_h1()();
		||
		call run_XOR_ANN_h2()();
		||
		call run_XOR_ANN_h3()();
	];
	pause;
		emit y(run_XOR_ANN_o1());
	pause;
end
		\end{lstlisting}
	\end{subfigure}
	\caption{Multi-periodic neurons, $WCRT = \left(i_1 \odot i_2\right) \oplus \left(h_1 \odot h_2 \odot h_3\right) \oplus o_1$}
	\label{fig:tca-neurons}
\end{subfigure}
\caption{Different arrangements for XOR \ac{MLP}}
\label{fig:tcas-xor}
\end{figure}%\pr{This Figure could be split and each aspect properly explained in the text.}

As mentioned, using the approach in Figure~\ref{fig:tca-bb-n} to implement complex \acp{ANN} i.e. \acp{CNN} may lead to implementations that violate their timing requirements. 
The synchronous approach, however, enables efficient implementations via compositional modelling.
Esterel supports this via statements such as \texttt{pause} that help to create smaller ticks, \texttt{;} for indicating sequence operation, and \texttt{||}, which enables the modelling of concurrency. 

For example, the \ac{MLP} in Figure~\ref{fig:tca-bb-n} can also be
compositionally modelled by splitting the neural function $\eta$
into its constituent layers $l_i$, $l_h$, and $l_o$. 
While this maintains the overall functionality of the network, it
reduces the \ac{WCRT} by making the network multi-periodic, where each
period is shorter. 
As previously mentioned, the \ac{ESS} case study in Section~\ref{sec:results} examines the benefits of this approach.

%This has no major effect in Figure~\ref{fig:tca-bb}, where the network is still running in a single tick, as function $f$ will call these layers internally anyway.
%However, if \texttt{pause} statements are introduced between each layer, then the timing of the program changes significantly: 
%the \ac{WCRT} of the program will decrease, and the execution becomes `multi-cyclic'.
%Now, multiple \textit{ticks} are required to complete the execution of the neural network, but the program's responsiveness can be increased.

\begin{prop}
	%As long as a neural network's layers are still executed in
	%the correct order, the output will be correct after the last
	%layer is completed.
	The mono-periodic implementation of the XOR \ac{MLP} is functionally
	equivalent to the multi-periodic implementation.
\end{prop}

\begin{proof}
	For any input, output vectors $\mathbf{i, o}$ respectively, 
	$\eta\left(\mathbf{i}\right) = l_o\left(
	l_h\left( l_i\left( \mathbf{i} \right) \right) \right) =
	\mathbf{o}$
	%The proof of this follows intuitively as the operation of $\nu$ already recursively calls each layer $l \in L$ of neurons during execution --- we just remove one layer of indirection.
\end{proof}

\ignore{
	\begin{example}
		In the XOR \ac{MLP}, there are three layers $L = \{l_i, l_h, l_o\}$. 
		During execution, $\eta$ will invoke each layer in order as it executes each neuron. 
		We can make this explicit by calling layers as functions. 
		In this case, $\eta\left(i\right) = l_o\left(
		l_h\left( l_i\left( i \right) \right) \right) = o$
	\end{example} %\pr{While I agree that this result should hold, it is
	%unclear how function composition is actually happening in the
	%code. Code only provides a sequence of function calls without
	%showing the composition!}
}

\ignore{
	For example, a \ac{MLP} $n$ may be compositionally modelled using its constituent layers $l_i$, $l_h$, and $l_o$, as shown in Figure~\ref{fig:tca-bb} (b), (c) and (d).
	\pr{Explain each approach and the trade off. What is the distinction between approach (a) and (b)? Also, could we have just one theorem to 
		state that all approaches are equivalent i.e. they produce the same output albeit in different ticks.}
	
	\begin{lemma}
		A single-tick `layer by layer' execution of functions $l \in L$ the same output as executing function $n$ if all layers are executed in order.
	\end{lemma} 
}




\ignore{
	This has not reduced the \ac{WCRT} of the system, as $n$ has the same \ac{WCRT} as the sum of the three layers that make it up.
	However, if the layers are split over \emph{multiple ticks}, then the system \ac{WCRT} can reduce.
	
	An example of this multi-tick `layer by layer' approach is presented in Figure~\ref{fig:tca-layers}.
	
	\begin{lemma}
		A multi-tick `layer by layer' execution of layers $l \in L \in n$ will give the same output as executing network $n$ completely, iff all layers are executed in order and once a number of ticks equivalent to the number of layers has elapsed. I.e. the number of logical ticks $= |L|$.
	\end{lemma}
	The proof of this is \todo{?}
}

Observe that the response time of the multi-periodic implementation is
$3 \times T_2$, where $T_2$ is the \ac{WCRT} of the multi-periodic
version. In contrast, the response time as well as the reaction time
of the mono-periodic version is always its \ac{WCRT} $T_1$. However, it is
expected that $T_2 << T_1$.
%It is important to note that a multi-tick execution of an \ac{SNN} will not speed up the generation of outputs of the overall network --- as now, multiple ticks must be completed before a given output will be ready. 
%However, the length of each individual tick is reduced, which means that overall the system interactivity can be increased.

Like the breaking up of the function $\eta$ into its component
layers, we can also break each layer $l$ into its component neurons.
This option is presented in Figure~\ref{fig:tca-neurons}. 
Unlike in Figure~\ref{fig:tca-bb}, neurons are able to operate concurrently within each layer. 
This could provide benefits when considering multi-core
implementations such as in \cite{yuan2011compiling}.
\ignore{ however, for single-core single-threaded execution (the focus of this paper), this approach does not considerably alter tick length.} 

A summary of these approaches is presented in Table~\ref{tbl:sann-approaches}.
%we separated each neuron call with \texttt{pause} statements --- i.e. called each neuron in its own tick. 
\begin{table}[H]
	\centering
	\caption{Summary of \ac{SNN} approaches}
	\label{tbl:sann-approaches}
	\begin{tabular}{|l|l|l|l|l|}
		\hline
		Arrangement    & Figures & WCRT & Cycles & Response Time   \\ \hline
		Mono-periodic  & \ref{fig:tca-bb-n},\ref{fig:tca-bb} & High & 1 & == WCRT              \\
		Multi-periodic & \ref{fig:tca-layers},\ref{fig:tca-neurons}& Low  & \textgreater{}1 & \textgreater{}\textgreater{}WCRT \\ \hline
	\end{tabular}
\end{table}


\ignore{ 
	\pr{I suggest we use consistent terminology i.e. mono-cyclic TCA and multi-cyclic TCA and their corresponding Esterel implementation. Ideally, the proof of the theorem 
		can be done using induction. Consider a mono-cyclic implementation and multi-cyclic implementation of length $k$.
		Base case -- first time outputs after 1 tick and k ticks are same. Hypothesis -- outputs after $m$ executions are equivalent i.e. First network executed for 
		$m$ ticks while the second network executed for $k\ times m$ ticks and produced the same output starting from the same input. Proof: to show that this holds for $m+1$ executions.}
}

%\subsection{Implementing `Black-Box' \ac{SNN} in Esterel}

%The basic programming unit in Esterel is a \textit{module}. 
%Each module contains an interface declaration, followed by a body of executable statements.
%Variables are known as \textit{signals}, and may consist of a \textit{status} and a \textit{value} component. 
%Signals with only a status signals are denoted \textit{pure}, while those that contain values are instead denoted \textit{valued}.
%Sections of code may be separated with either the `\code{;}' or `\code{||}' operator, which denote sequencing and synchronous concurrency respectively.
%In order to perform manipulation of valued signals, Esterel can call external functions (provided in supplementary C files).

%Presented in Listing~\ref{lst:blackbox} and Listing~\ref{lst:blackbox-composition} are the simplest way of implementing neural networks in Esterel, already seen in the AI-BRO example.
%As can be seen, entire \acp{ANN} can be composed together and are simply called as external function calls, and will run each tick. 
%Networks can be composed together
%This can be considered the `bare minimum' to qualify as a \ac{SNN}, as the network is now running synchronously with its environment.

%Even though this implementation is quite trivial, it can still gives a strong advantage over a general-purpose implementation: when considering the methodology for composing multiple \acp{ANN}. 
%For example, consider a system that needs three \acp{ANN}, each performing a different task (as is the case in our AI-BRO example, in Section~\ref{sec:motivating-example}). 

%We can compose these multiple \acp{ANN} synchronously in Esterel. 
%As an example of this, consider an extension of our XOR network that computes $y = \big((x_1$ \textit{xor} $x_2)$ \textit{xor} $(x_3$ \textit{xor} $x_4)\big)$ (i.e. $y$ is $true$ when an odd number of inputs are $true$).
%We can compose this together still using the `black-box' methodology, as presented in Listing~\ref{lst:blackbox-composition}.
%Here, it can be seen that the first two \acp{ANN} can be run concurrently with one another, and the third \ac{ANN} can only be run when the first two are complete.


