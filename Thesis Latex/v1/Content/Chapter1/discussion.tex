\section{Discussion}
\label{sec:conclusions}

Neural networks are being used in many \acf{CPS} to influence
real-time decisions with safety implications. Consequently, there has
been a convergence of AI and formal methods to ensure that such
applications operate safely at all times~\cite{seshia2016towards}. This has led to the development of new
techniques for functional verification of AI-based \ac{CPS}. However,
the issue of timing verification has received scant attention. 

This thesis, for the first time, develops neural networks with timed
semantics called \acfp{SNN}. Our approach develops periodic networks,
where the response time of a network may be between $1$-tick (called
\emph{mono-periodic} networks) to $n$-ticks (called
\emph{multi-periodic} networks). We propose timing semantics of such
networks using \ac{WCRT}-algebra~\cite{wang2017timing}. Multi-periodic networks are
introduced to reduce the system reaction time, while keeping the
response time within the specified deadline (see the \texttt{ESS} case
study in Section~\ref{sec:results}). 

For complex CPS, we propose meta neural networks for designing concurrent applications where the output from
one network influences other networks. We provide several alternative
architectures and study such systems in
Section~\ref{sec:results}. Overall, this thesis develops a new approach
for the design of time predictable \acf{CPS} involving interacting 
AI modules.

As this is the first work on synchronous neural networks,
there are some limitations and hence several avenues
for future research exist. First, all benchmarks use off-line learning. We are already
working on developing techniques for on-line learning in Esterel. A
second limitation is that we are relying on ``compiling away'' Esterel
concurrency, which is not ideal for many-core processors. We will develop parallel implementations
of \acp{SNN} in the near future. 
Additionally, we developed \acp{SNN} to ensure timing correctness. 
\ignore{However, synchronous semantics could also aid in the
validation of functional correctness for \acp{ANN}, due to their
deterministic and reactive nature. This aspect will be examined in the future.}
Finally, we could have considered
data-flow synchronous languages like Lustre~\cite{benveniste2003synchronous} for
\acp{SNN}. However, we selected Esterel due to the nature of \ac{CPS}
applications, which combine data and control-flow, where the Esterel
style seems more natural.