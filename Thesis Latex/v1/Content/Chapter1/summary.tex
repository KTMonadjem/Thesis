\section{Summary}
This chapter introduced the concept of \acfp{SNN} and their timing properties.

In Section~\ref{sec:motivating-example}, \acp{SNN} and their behaviour is introduced. 
Additionally, a motivating example is given with it's Esterel implementation.

Section~\ref{sec:wcrt} introduces the methods by which timing analysis is done on \acp{SNN} and the algebra involved with this.
This chapter also gives the first formal definition of a \ac{SNN} and discusses the periodicity of these \acp{SNN}.

The next section, Section~\ref{sec:concurrent-sann}, introduces the concept of \acfp{MNN} and provides formal definitions for them, as well as providing examples of the structure of the introduces \acp{MNN}.

A new Python compiler for creating \acp{MNN} from Keras is introduced in Section~\ref{sec:mnn2c} and its results are presented.

The final section of this chapter presents the results of the work done on \acp{SNN} and \acp{MNN} for this chapter.
These results are discussed and analysed, with conclusions drawn in the following chapter.
