\section{Timing Analysis of \acfp{SNN}}
\label{sec:wcrt}

Time in a synchronous program progresses in logical discrete instants called \emph{ticks}. 
In each tick, the environment is first sampled, then computation takes places, and finally the results are emitted.
In order for safety-critical \ac{CPS}, such as our \texttt{AI-BRO}, to be verified, we need to ensure that the time taken for the execution of any given tick is less than their environmental deadlines.
This necessitates the computation of the worst-case tick length via static timing analysis --- also known as \acf{WCRT} analysis~\cite{roop2009tight}.

\ac{WCRT} analysis needs two things: (1) an architectural model of the
underlying processor architecture; and (2) a \acf{CFG}~\cite{wilhelm2008worst}  representation of
binary program execution over the architectural model. The analysis
process then computes the longest path through the \ac{CFG}.
%For any program, static timing analysis involves first the derivation of a \acf{CFG}~\cite{wilhelm2008worst} which captures the possible execution traces.
%Then, the \ac{CFG} must be annotated with the time taken to execute each node, so that a worst-case path can be derived.
While general purpose programs can feature complications which make this process difficult, Esterel is ideally suited to derivation of \acp{CFG} due to inherent features:  (1) all loops in Esterel are bounded, as there is a need for at least one \texttt{pause} statement
within every loop; and (2) compilers ``compile-away'' the logical
concurrency to produce sequential code. However, as can be seen in
Listing~\ref{lst:ai-bro}, 
Esterel programs can also invoke arbitrary C-functions, which may use
features such as function pointers, unbounded loops, and external interrupts. 
To this end, we provide a set of guidelines and \ac{ANN}
implementation libraries in C,
which restrict these features to make each C function call time
analysable. 

%\subsection{Guidelines}
In this paper, we are using the open-source Platin
tool~\cite{compiler:platin:kps15} for \ac{WCRT}
analysis. Consequently, we have to consider some tool-specific
restrictions to avoid the use of generic features in C that are not
suitable for static analysis, such as system calls, function pointers, interrupt-driven flow, or unbounded loops~\cite{RTOSWCET}.
Hence, the restrictions considered in this paper are: (1) all neural
network functions are coded as bounded \texttt{for} loops; (2) all
memory is statically allocated; (3) variable-length array iterator
functions are coded using preprocessor macros; and (4) the final binary is executed `bare-metal', requiring no
RTOS or OS support, thus avoiding scheduling dependencies
that may be difficult to statically analyse.
\ignore{
	\begin{itemize}
		\item All neural network functions are coded as bounded \texttt{for} loops.
		\item All memory is statically allocated.
		\item Variable-length array iterator functions are coded using preprocessor macros.
		\item The final binary is executed `bare-metal', requiring no
		RTOS or OS support, thus avoiding scheduling dependencies
		that may be difficult to statically analyse.
	\end{itemize}
}

Annotating a \ac{CFG} with execution times is no simple task for many architectures.
Enhancements such as out-of-order execution, multi-level caches, and deep pipelines can all cause complications when considering the time taken to execute program code.
For instance, in~\cite{AirplaneWcet}, Ferdinand et. al. demonstrate the difficulties of determining the worst-case time in arguably simple avionics code running on a Motorola ColdFire 5307.
To avoid these complications, we rely on the Patmos~\cite{patmos:ppes2011} architecture, which is a simple RISC pipeline optimised for static timing analysis. 
Here, caches have simple replacement policies, and it also has a time-predictable memory hierarchy.
It relies on an adaptation of the LLVM compiler that targets the Patmos instruction set.
This simplifies the annotation of the nodes of the \ac{CFG} with
timing costs, thus allowing for a straightforward static analysis of any
\ac{SNN}. Consequently, this work is restricted to the single-core
Patmos architecture and does not examine execution of \acp{SNN} on
other platform types, such as Intel processors and GPUs.
%\changed{Consequently, this work is restricted to the single-core Patmos architecture and does not examine execution of \acp{SNN} on other platform types, such as Intel processors or GPUs}.

%The expense of this is program size, and designers can expect slightly increased program memory usage as a result. 

%When it comes to the annotation of these nodes with their execution times, architectural considerations must be taken into account, for instance memory access times and pipelined execution of instructions.


%\subsection{Patmos}


%We then use the open-source academic timing analysis tool called Platin~\cite{compiler:platin:kps15}, to derive our execution times.


\ignore{
	\begin{example}
		Consider the case of our AI-BRO example in Listing~\ref{lst:ai-bro}. 
		We can express this Esterel program using the \ac{CFG} presented in Figure~\ref{fig:cfg-aibro}.
		\begin{figure}
			\centering
			\begin{tikzpicture}[->,>=stealth',shorten >=1pt,auto,
	node distance=2.5cm,
	semithick,scale=0.8, transform shape,scale=0.6, font=\huge]
	
	\tikzstyle{every state}=[rectangle,
	minimum height = 1cm, text width=3.5cm, text centered, 
	fill=blue!20,draw=none,text=black, draw,line width=0.3mm, font=\LARGE]
	
	%start state
	\node[state]
	(init) 
	{$f$: \\init, \textit{fork}}; 
	
	\node[state]
	(awaitA) [below of=init] 
	{$a_1$: \\await \texttt{A}};
	
	\node[state]
	(frontA) [below of=awaitA] 
	{$a_2$: \\frontSensors()};
	
	\node[state]
	(awaitB) [below of=frontA]
	{$b_1$: \\await \texttt{B}};
	
	\node[state]
	(sideB) [below of=awaitB]
	{$b_2$: \\sideSensors()};
	
	\node[state]
	(awaitJoin) [below of=sideB] 
	{$j$: \\ \textit{await join}};
	
	\node[state]
	(decisionC) [below of=awaitJoin] 
	{$c$: \\runDecision()};
	%\node[draw=none]
	%(inv) [below of=init]
	%{};
	
	%\node[draw=none]
	%(inv2) [below of=inv]
	%{};
		
	

	
	\path[->] (init) edge (awaitA);
	\path[->] (awaitA) edge (frontA);
	\path[->] (frontA) edge (awaitB);
	\path[->] (awaitB) edge (sideB);
	\path[->] (sideB) edge (awaitJoin);
	\path[->] (awaitJoin) edge (decisionC);
	
	\draw[->] (decisionC.south) 
		|- ($(decisionC.south) + (0, -1cm)$) 
		-| ($(decisionC.east) + (2.5cm, 0)$) 
		|- ($(init.east) + (0, 1.5cm)$) 
		-| (init.north);
		
	\draw[->] (awaitA.west) 
		-| ($(awaitA.west) + (-1cm, 0)$)
		|- ($(awaitB.west) + (-1cm, 0)$)
		%|- ($(awaitA.160) + (0, 0.75cm)$) 
		|- (awaitB.west);
		
	\draw[->] (awaitA.west) 
		-| ($(awaitA.west) + (-1cm, 0)$)
		|- ($(awaitB.west) + (-1cm, 0)$)
		%|- ($(awaitA.160) + (0, 0.75cm)$) 
		|- (awaitJoin.west);
		
	\draw[->] (awaitB.east) 
		|- ($(awaitB.east) + (2.5cm, 0)$);
		
	\draw[->] (awaitJoin.east) 
	|- ($(awaitJoin.east) + (2.5cm, 0)$);
	
	%\draw[->] (decisionC.south) 
	%	|- ($(decisionC.south) + (0, -1cm)$) 
	%	-| ($(decisionC.east) + (4.5cm, 0)$) 
	%	|- ($(init.east) + (0, 1.5cm)$) 
	%	-| (init.north);
	
	%\draw[->] (awaitA.200) 
	%	|- ($(awaitA.200) + (0, -0.75cm)$) 
	%	-| ($(awaitA.200) + (-2cm, 0)$)
	%	|- ($(awaitA.160) + (0, 0.75cm)$) 
	%	-| (awaitA.160);
		
	%\draw[->] (awaitB.340) 
	%|- ($(awaitB.340) + (0, -0.75cm)$) 
	%-| ($(awaitB.340) + (2cm, 0)$)
	%|- ($(awaitB.20) + (0, 0.75cm)$) 
	%-| (awaitB.20);
	
	%\draw[->] (awaitJoin.195) 
	%|- ($(awaitJoin.195) + (0, -0.75cm)$) 
	%-| ($(awaitJoin.195) + (-1.5cm, 0)$)
	%|- ($(awaitJoin.165) + (0, 0.75cm)$) 
	%-| (awaitJoin.165);
	
	
\end{tikzpicture}

			\caption{Single-thread AI-BRO \ac{CFG}}
			\label{fig:cfg-aibro}
		\end{figure}
	\end{example} \pr{This figure is somewhat non-standard. There are no
		conditional nodes. I suggest you examine the syntax of CFGs
		presented in WCET papers and follow these.}
}

%Fortunately, it is straightforward to convert the synchronous C code generated by Esterel into its equivalent control flow graph.
%Then, as long as the external C function calls that it will call are also predictable, and if the code is run on a time-predictable architecture such as Patmos~\cite{patmos:ppes2011}, the code can be %converted into an equivalent \ac{TCFG}.


%This section discusses the WCRT analysis of SNNs.
%Introduce the WCET problem and then introduce WCRT.
%Intermediate format, timing model, the PATMOS core.
%Coding guidelines for time analysable code. Steps to improve the 
%WCRT. These can be the various subsections.


%When considering normal C code, such as in our external function calls, it can be extremely difficult to construct \ac{CFG}.
%This is because it can be difficult to statically an
%Constructing a \ac{CFG} for a normal general purpose program can be very difficult, especially if it relies on complex software constructs such as system calls, function pointers, interrupts, or unbounded loops.
%This is because the construction of the \ac{CFG} becomes extremely difficult in the face of large branching potentials or unknown jump destinations.


\subsection{\ac{WCRT} algebra}
\label{sec:wcrt-algebra}

Though Esterel is amenable to timing analysis, it still suffers from state space explosion when considering large programs. 
Analysing large sequential code generated from Esterel compilers that ignore \emph{infeasible state space} 
produce large overestimates. \ac{WCRT} algebra, presented in \cite{wang2017timing}, presents a solution to this problem: large synchronous programs can be expressed as 
networks of simpler \ac{TCA}, which have equivalent timing properties to their \ac{TCFG} counterparts. Considering this, we
seek to use this framework to develop a compositional timing semantics of \acp{SNN} and their compositions.

We consider a discrete max-plus structure over
natural numbers $(\Nat_{\infty}, \oplus, \odot, \zero, \one)$ where:
\begin{itemize}
	\item $\Nat_{\infty} \dfs \Nat \cup \{ -\infty, +\infty \}$
	\item  $\oplus$ and $\odot$ are commutative and associate operators that 
	denote the maximum and addition respectively on $\Nat_{\infty}$. 
	\item Neutral elements are $\zero \dfs -\infty$ and $\one \dfs 0$,
	respectively, i.e., $x \oplus \zero = x$ and $x \odot \one = x$. 
\end{itemize}

In particular, $-\infty \odot +\infty = -\infty$. Addition
$\odot$ distributes over max $\oplus$, i.e.,
$x \odot \left(y \oplus z\right) = x + \maxd\left(y, z\right) = \maxd\left(x + y, x + z\right) =
\left(x \odot y\right) \oplus \left(x \odot z\right)$. However, $\oplus$ does not distribute
over $\odot$, for instance, $9 \oplus \left(5 \odot 7\right) = \maxd\left(9, 5 + 7\right) = 12$
while $\left(9 \oplus 5\right) \odot \left(9 \oplus 7\right) = \maxd\left(9, 5\right) + \maxd\left(9, 7\right) = 18$.
This induces on $\Nat_{\infty}$ a (commutative, idempotent) semi-ring.
We denote $\mathcal{W}$ as the set of formulae in this algebra.

We use an automaton called \ac{TCA} to capture the timing behaviour of any \ac{SNN}.
These are mono-cyclic automata, where every state of the automaton is labelled by a
WCRT formula that associates the state with its worst case timing cost.
We start by defining \ac{TCA} using Definition~\ref{def:tca}
%---------------------------------

\begin{definition}
	A \emph{mono-cyclic tick cost automaton}, here after referred to as tick cost automaton (\ac{TCA}) 
	is %a tuple 
	$M = \langle Q, q_0, \rightarrow, \mathcal{W}, L
	\rangle$, where $Q$ is a finite set of \emph{states} with a
	distinguished \emph{entry} state $q_0 \in Q$ and $L: Q \rightarrow \mathcal{W}$ is
	a labeling function that labels every state with its associated timing cost by a formula in $\mathcal{W}$.
	The \emph{transition relation}
	${\rightarrow} \subseteq Q \times Q$ satisfying the following:
	
	\begin{enumerate}
		\item From any state $q_i \in Q$ there is at most one transition to a successor state $q_j \in Q$.
		\item The transitions are such that they form a mono-cycle that starts in $q_0$ and and ends in a 
		transition from some state $q_{n-1} \rightarrow q_0$, resulting in a
		cycle of length $n \in \Nat_{>0}$.
		\item A \ac{TCA} whose cycle length is $1$ is termed
		\emph{mono-periodic}, while those with cycle length $>1$ are called
		\emph{multi-periodic}.
		\item The \emph{reaction time} of a \ac{TCA} is its tick-length, which is
		always fixed to its \ac{WCRT}. 
		\item The \emph{response time} of a \ac{TCA} is the time taken for the
		corresponding \ac{SNN} to produce the outputs given inputs. 
		The response time of a TCA is given by the formula $n \times WCRT$,
		where $n$ is the cycle length.
		%\item The above conditions ensure that a \ac{TCA} is a mono-cyclic automaton.
	\end{enumerate}
	\label{def:tca}
\end{definition}

%---------------------------------
\ignore{
	\pr{Hammond: I have defined both reaction and response time of a
		TCA. We should use these      n our benchmarks, if possible.}
	\begin{example}
		$15 \oplus \left(5 \odot 10\right) = 15$, and $\left(15 \oplus 5\right) \odot \left(15 \oplus 10\right) = 30$.
	\end{example}
	
	\begin{example}
		Consider the program in Listing~\ref{lst:ai-bro}. 
		This had two concurrent threads, which are compiled away to produce sequential code.
		We can express this network's \ac{WCRT} as \\
		$WCRT = \left( a_{await} \odot a_{run} \odot b_{await} \odot b_{run} \odot c_{awaitJoin} \odot c_{run} \odot c_{emit} \right)$, as this is the longest potential path through the system (if we receive both \texttt{A} and \texttt{B} in the same tick).
	\end{example} \pr{Hammond -- since the figure is removed, this formula
		doesn't make any sense now. Please check.}}
%$x \oplus y =$ $MAX(x,y)$ 
%$x \odot y =$ $x + y$ 