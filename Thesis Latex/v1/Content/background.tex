\section{Artificial Neural Networks}
\subsection{Machine Learning}
\acp{ANN} were originally proposed to mimic the functioning of  biological neural networks~\cite{kohonen1988introduction}, which produce recurrent spatio temporal patterns~\cite{rolston2007precisely}. 
Similar timed activity of neurons in the cerebellum has been reported in~\cite{bullock1994neural, moore1989adaptively}. 
A number of types of \ac{NN} which mimic their biological counterparts exist, varying in complexity and accuracy, including the \ac{SNN}~\cite{izhikevich2003spiking,maass1997spiking}, which was designed to model the brain and has been demonstrated to be periodic and run with discrete time intervals when implemented in software. 

\subsection{Structure of an Artificial Neural Network}
Most \acp{ANN} do not feature such complex models like those of spiking neural networks, as they are more difficult to use, implement, and train. 
Instead, they rely on simpler networks, which can be considered as \emph{un-timed non-linear} functions, where the outputs change relative to the inputs, but the timing of the change is not precisely defined. 
An example of such a network is provided in Figure~\ref{fig:mlp-ann}, which is using neurons defined in Figure~\ref{fig:artificial-neuron}. 
This is a type of \ac{NN} known as an \acf{MLP}~\cite{yegnanarayana1994artificial}.

Specialised neural networks, called \acfp{RNN}~\cite{medsker2001recurrent}, were introduced to classify temporal sequences. 
These operate in a step by step manner, where the operation in the current time step relies on the context from some previous step. Thus, \acp{RNN} may be viewed as a periodic networks, whosemperiod is one. 
However, the use of such networks in \ac{CPS} is yet to be thoroughly investigated. 
Moreover, \acp{RNN} and their compositions are not formalised especially from the point of view of designing timed AI systems used in \ac{CPS}.  

\begin{figure}
	\centering
	\scalebox{0.8}{\input{fig/mlp-ann.tex}}
	\caption{Example \ac{MLP} \ac{ANN}.	\label{fig:mlp-ann}}
\end{figure}
\begin{figure}
	\centering
	\scalebox{0.8}{\input{fig/neuron.tex}}
	\caption{A model of an artificial neuron. \label{fig:artificial-neuron}}
\end{figure}

\acp{ANN} are being increasingly used as controllers in \acp{CPS} due to their ability to learn data relationships in ways that are difficult to replicate~\cite{ANNSafety2007}. 
\acp{ANN} can deal with novel inputs to the system and are able to outperform other forms of \ac{AI} at computational efficiency, pattern recognition, function approximation and image identification~\cite{AIComp2016, AIComp2017}. 
However, it can be very difficult to ensure the safety of a system involving \acp{ANN}~\cite{ANNSafety2007, ANNSafety2018}.

\subsection{Safety of Artificial Neural Networks}
In order for an \ac{ANN} to be used in any capacity within a system where safety is critical, it should undergo rigorous and thorough validation, verification, and testing procedures to ensure that they it is sufficiently safe for its target system~\cite{scann, ANNSafetyLifecycle2003}. 

While considerable research effort is starting in the direction of formal verification of \ac{AI}-based \ac{CPS}~\cite{seshia2016towards, russell2015}, the issue of timing verification has received scant attention. 
Like the challenges involving functional verification, timing verification of AI-based  \ac{CPS} poses considerable challenges due to: (1) real-time \ac{AI} systems could involve many concurrent and interacting \ac{AI} modules, which need deterministic composition for safety; (2) \ac{AI} modules are usually developed as untimed systems and the reactive nature of AI algorithms used in CPS are not carefully studied; and (3) \acf{WCET} analysis~\cite{wilhelm2008worst} of \ac{AI}-based \ac{CPS} has received scant attention.

Definitions for this safety vary, but Kurd et. al.~\cite{EstSafeCriteria2003} provide a generalisation: safe \acp{ANN} can be defined as those that:
\begin{itemize}
	\item tolerate faults and inconsistencies in their inputs,
	\item do not create hazardous outputs,
	\item behave in a predictable and repeatable manner,
	\item and are trained on clean, reliable data. 
\end{itemize}

To achieve these properties, there exist safety measures such as risk management systems that span the entire development process of the \ac{ANN}~\cite{ANNDevModel1999} and standards with which \acp{ANN} can be certified before they are used in systems where safety is critical~\cite{SCANNStandard}. 
These techniques are primarily \textit{proactive} in nature, producing \acp{ANN} that are classified as \textit{safe enough} for their role. 

However, as \acp{ANN} become larger and more full-featured, they  become harder to statically analyse.
Problematic situations can arise when an \ac{ANN} exhibits unexpected behaviour that the system is unable to safely respond to, and in \ac{CPS} these situations can be life threatening.

\subsection{Static Verification of \acp{ANN}}
There are a variety of pre-existing methods for statically checking the correctness of autonomous (i.e. artificially intelligent) systems.
For instance, model checking on systems that use timed automata~\cite{timed-enf-autonomous}.
Okano et. al. explore the concept of model checking of autonomous systems that use timed automata~\cite{timed-enf-autonomous}. 
This technique is a static technique and cannot be applied to autonomous \ac{ANN} systems, where the behaviour of the \ac{ANN} controller cannot be defined by an automaton.
For an autonomous system that includes at least one \ac{ANN} in its controller, novel techniques are required to guarantee the safety properties of the \acp{ANN}
However, \acp{ANN} are not usually able to be simplified to simple automata.

Verification of \ac{ANN}, specifically, Deep Neural Networks, can be performed for certain properties (such as robustness) using \ac{SMT}~\cite{Gehr2018AI2SA,reluplex,DeepANNverify}.
This is useful, because the robustness of a deep \ac{ANN} is critical property to its safety.
A robust \ac{ANN} is one that will provide consistently accurate outputs even when the input to the \ac{ANN} is noisy, incorrectly coloured or otherwise distorted. 
However, this approach is not flawless. 
\ac{SMT} has issues with scale: as the \acp{ANN} to analyse become larger, analysis time grows exponentially~\cite{Gehr2018AI2SA}.
Ergo, they are less efficient on larger deep \acp{ANN}.
In addition, they require the deep \acp{ANN} to fulfil some specific properties, such as specific activation functions and specific \ac{ANN} variants, i.e. \cite{Gehr2018AI2SA} only allows \acp{CNN} and \acp{MLP} with \ac{ReLU} activation functions.
This limits flexibility, as each \ac{ANN} must be designed around these restrictions, thus limiting properties of the \ac{ANN}, such as its activation function and size, could result in an \ac{ANN} that is inefficient, not robust, slow, etc.

Manual testing is a simple method to check for the correctness of an \ac{ANN}-based system. 
However, this is a time-consuming and error-prone process.
It is very difficult to ensure that tests have acceptable coverage of all possible situations~\cite{ANN-test}.
Test cases, validation and verification of \acp{ANN} in \acp{SCS} systems can only go so far; test data is not unlimited, time is a resource, verification is not 100\% accurate and humans make errors. 

Finally, no matter the chosen methodology, as \acp{ANN} increase in size and complexity, verification and validation of these networks becomes increasingly more difficult to achieve~\cite{Gehr2018AI2SA}.


\section{Synchronous Languages}
\subsection{Synchronous Semantics}
Briefly explain synchronous semantics: what is synchronous programming and how does it work?

\subsection{Safety of Synchronous Languages}
What guarantees does synchronous programming provide?
Determinism
Causality
Easy to formalise
Etc

\subsection{Timing Correctness}
Time predictability.

\subsection{Functional Correctness}
Runtime Enforcement






\section{\acf{RE} of Autonomous Systems}
%% TALK ABOUT RA AND RE

\subsection{Runtime Assurance}
% Runtime assurance
\ac{RA} is the ability to to ensure that a system operates safely, even when the system contains components that are not sufficiently reliable or verified~\cite{rta-cps}. 
Thus, \ac{RA} can be used to bound unpredictable or unsafe behaviour in a target system. 
\ac{RA} has been used as a reliability and fail-safe tool for some time, for instance in \acp{OS}, intrusion detection, overcurrent breakers and flight controllers. 
A system that uses \ac{RA} shifts the burden of testing and analysing system parameters from comprehensive off-line verification methods, to a simpler assurance mechanism~\cite{rta-cps}. 

\subsection{Runtime Enforcement}
% Runtime enforcement
\ac{RE} is a subset of \ac{RA} that focuses on formal semantics and blocking, delaying, modifying and/or re-ordering of events in a system. 
Processes that are deemed unsafe can be monitored by an enforcer at runtime to ensure that they obey desired policies and remain in a safe state at all times~\cite{theoryRE}. 
\ac{SA} formally express run-time properties that can be monitored in one direction only (e.g. outputs only)~\cite{enfsafepol}. 
Edit automata are a type of \ac{SA} that can edit, suppress or insert events~\cite{editautomata}. 
Safety Automata (or \ac{DTA}) have been proposed that can edit \textit{bi-directional} events at runtime~\cite{recps}. 
They were designed for reactive \acp{CPS} demonstrated in a pacemaker environment~\cite{recps}. 

\subsection{Safety (Timed) Automata}
These are used in our run-time enforcement techniques. Discuss these and how they work.

\subsection{Runtime Enforcement of \acp{ANN}}
% Write about exisiting autonomous RE
The idea of \ac{RE} of autonomous systems has been looked into previously. 
De Niz et. al. propose a type of \ac{RE} they term temporal enforcement, which ensures that the system controller meets timing deadlines where outputs are concerned~\cite{safe-enforce-auto}. 
While this shares similarities with the work in this paper, their work does not expand to cover \acp{ANN}, and does not propose the use of \ac{RE} for anything other than meeting timing deadlines.
Aniculaesei et. al. propose static formal verification and runtime monitoring of autonomous, robotic systems to prevent physical collisions during system execution~\cite{runtime-monitor}.
While this looks at the enforcement of system outputs, the inputs are not monitored and the timing deadlines of the system are not investigated. 
Additionally, the case study involves a robot controlled by an automaton, not a highly complex \ac{AI} such as an \ac{ANN}.





\section{Summary}
Summary of above.

\todo{IMAGES PLEASE}