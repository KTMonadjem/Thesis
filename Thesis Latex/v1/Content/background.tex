\section{Artificial Neural Networks}
\subsection{Machine Learning}
\acp{ANN} were originally proposed to mimic the functioning of  biological neural networks~\cite{kohonen1988introduction}, which produce recurrent spatio temporal patterns~\cite{rolston2007precisely}. 
Similar timed activity of neurons in the cerebellum has been reported in~\cite{bullock1994neural, moore1989adaptively}. 
A number of types of \ac{NN} which mimic their biological counterparts exist, varying in complexity and accuracy, including the \ac{SNN}~\cite{izhikevich2003spiking,maass1997spiking}, which was designed to model the brain and has been demonstrated to be periodic and run with discrete time intervals when implemented in software. 

\subsection{Structure of an Artificial Neural Network}
Most \acp{ANN} do not feature such complex models like those of spiking neural networks, as they are more difficult to use, implement, and train. 
Instead, they rely on simpler networks, which can be considered as \emph{un-timed non-linear} functions, where the outputs change relative to the inputs, but the timing of the change is not precisely defined. 
An example of such a network is provided in Figure~\ref{fig:mlp-ann}, which is using neurons defined in Figure~\ref{fig:artificial-neuron}. 
This is a type of \ac{NN} known as an \acf{MLP}~\cite{yegnanarayana1994artificial}.

Specialised neural networks, called \acfp{RNN}~\cite{medsker2001recurrent}, were introduced to classify temporal sequences. 
These operate in a step by step manner, where the operation in the current time step relies on the context from some previous step. Thus, \acp{RNN} may be viewed as a periodic networks, whosemperiod is one. 
However, the use of such networks in \ac{CPS} is yet to be thoroughly investigated. 
Moreover, \acp{RNN} and their compositions are not formalised especially from the point of view of designing timed AI systems used in \ac{CPS}.  

\begin{figure}
	\centering
	\scalebox{0.8}{\def\layersep{2.25cm}
\def\numInp{4}
\def\numHid{5}
\def\numOut{3}
\begin{tikzpicture}[shorten >=1pt,->,draw=black!100, node distance=\layersep]
	\tikzstyle{every pin edge}=[<-,shorten <=1pt]
	\tikzstyle{neuron}=[circle,fill=black!25,minimum size=20pt,inner sep=0pt]
	\tikzstyle{input neuron}=[neuron, fill=white!100,draw=black];
	\tikzstyle{output neuron}=[neuron, fill=white!100,draw=black];
	\tikzstyle{hidden neuron}=[neuron, fill=white!100,draw=black];
	\tikzstyle{annot} = [text width=4em, text centered]
	
	% Draw the input layer nodes
	\foreach \name / \y in {1,...,\numInp}
	% This is the same as writing \foreach \name / \y in {1/1,2/2,3/3,4/4}
	\node[input neuron, pin=left:Input \y] (I-\name) at (0,-\y) {$i_\y$};
	
	% Draw the hidden layer nodes
	\foreach \name / \y in {1,...,\numHid}
	\path[yshift=0.5cm]
	node[hidden neuron] (H-\name) at (\layersep,-\y cm) {$h_\y$};
	
	% Draw the output layer nodes
	\foreach \name / \y in {1,...,\numOut}
	\node[output neuron, pin={[pin edge={->}]right:Output \y}] (O-\name) at (4.5,-0.25-\y) {$o_\y$};
		
	% Connect every node in the input layer with every node in the
	% hidden layer.
	\foreach \source in {1,...,\numInp}
	\foreach \dest in {1,...,\numHid}
	\path (I-\source) edge (H-\dest);
	
	% Connect every node in the hidden layer with the output layer
	\foreach \source in {1,...,\numHid}
	\foreach \dest in {1,...,\numOut}
	\path (H-\source) edge (O-\dest);
	
	% Annotate the layers
	\node[annot,above of=H-1, node distance=1cm] (hl) {\textit{Hidden Layer}};
	\node[annot,left of=hl] {\textit{Input Layer}};
	\node[annot,right of=hl] {\textit{Output Layer}};
\end{tikzpicture}}
	\caption{Example \ac{MLP} \ac{ANN}.	\label{fig:mlp-ann}}
\end{figure}
\begin{figure}
	\centering
	\scalebox{0.8}{\begin{tikzpicture}[
init/.style={
  draw,
  circle,
  inner sep=2pt,
  font=\Huge\itshape,
  join = by -latex
},
squa/.style={
  draw,
  inner sep=2pt,
  font=\Large,
  join = by -latex
},
start chain=2,node distance=10mm
]
\node[on chain=2] 
  (x2) {$x_2$};
\node[on chain=2,join=by o-latex] 
  {$w_2$};
\node[on chain=2,init] (sigma) 
  {$\displaystyle\Sigma$};
\node[on chain=2,squa,label=above:{\parbox{2cm}{\centering \textit{Activation \\ function}}}]   
  {$f$};
\node[on chain=2,label=above:\textit{Output},join=by -latex] 
  {$y$};
\begin{scope}[start chain=1]
\node[on chain=1] at (0,1cm) 
  (x1) {$x_1$};
\node[on chain=1,join=by o-latex] 
  (w1) {$w_1$};
\end{scope}
\begin{scope}[start chain=3]
\node at (0.9, -1.2cm) (dots) {...};
\node[on chain=3] at (0,-2cm) 
  (x3) {$x_n$};
\node[on chain=3,label=below:\textit{Weights},join=by o-latex] 
  (w3) {$w_n$};
\end{scope}
\node[label=above:\parbox{2cm}{\centering \textit{Bias} \\ $b$}] at (sigma|-w1) (b) {};

\draw[-latex] (w1) -- (sigma);
\draw[-latex] (w3) -- (sigma);
\draw[o-latex] (b) -- (sigma);

\draw[decorate,decoration={brace,mirror}] (x1.north west) -- node[left=10pt] {\textit{Inputs}} (x3.south west);
\end{tikzpicture}}
	\caption{A model of an artificial neuron. \label{fig:artificial-neuron}}
\end{figure}

\acp{ANN} are being increasingly used as controllers in \acp{CPS} due to their ability to learn data relationships in ways that are difficult to replicate~\cite{ANNSafety2007}. 
\acp{ANN} can deal with novel inputs to the system and are able to outperform other forms of \ac{AI} at computational efficiency, pattern recognition, function approximation and image identification~\cite{AIComp2016, AIComp2017}. 
However, it can be very difficult to ensure the safety of a system involving \acp{ANN}~\cite{ANNSafety2007, ANNSafety2018}.

\subsection{Safety of Artificial Neural Networks}
In order for an \ac{ANN} to be used in any capacity within a \acp{SCS}, it should undergo rigorous and thorough validation, verification, and testing procedures to ensure that they it is sufficiently safe for its target system~\cite{scann, ANNSafetyLifecycle2003}. 

While considerable research effort is starting in the direction of formal verification of \ac{AI}-based \ac{CPS}~\cite{seshia2016towards, russell2015}, the issue of timing verification has received scant attention. 
Like the challenges involving functional verification, timing verification of AI-based  \ac{CPS} poses considerable challenges due to: (1) real-time \ac{AI} systems could involve many concurrent and interacting \ac{AI} modules, which need deterministic composition for safety; (2) \ac{AI} modules are usually developed as untimed systems and the reactive nature of AI algorithms used in CPS are not carefully studied; and (3) \acf{WCET} analysis~\cite{wilhelm2008worst} of \ac{AI}-based \ac{CPS} has received scant attention.

Definitions for this safety vary, but Kurd et. al.~\cite{EstSafeCriteria2003} provide a generalisation: safe \acp{ANN} can be defined as those that:
\begin{itemize}
	\item tolerate faults and inconsistencies in their inputs,
	\item do not create hazardous outputs,
	\item behave in a predictable and repeatable manner,
	\item and are trained on clean, reliable data. 
\end{itemize}
%These properties are essential for an \ac{ANN} to run in any \ac{SCS}. 

To achieve these properties, there exist safety measures such as risk management systems that span the entire development process of the \ac{ANN}~\cite{ANNDevModel1999} and standards with which \acp{ANN} can be certified before they are used in \acp{SCS}~\cite{SCANNStandard}. 
These techniques are primarily \textit{proactive} in nature, producing \acp{ANN} that are classified as \textit{safe enough} for their role. 

However, as \acp{ANN} become larger and more full-featured, they  become harder to statically analyse.
Problematic situations can arise when an \ac{ANN} exhibits unexpected behaviour that the system is unable to safely respond to, and in \acp{SCS} these situations can be life threatening.







\section{Timing Analysis of Cyber-Physical Systems}
\subsection{Worst Case Execution Time}
Explain what WCET is in safety critical systems. Why is it important for deadlines to be met and how does this relate to WCET?

\subsection{Worst Case Reaction Time}
WCRT vs WCET. Throughput vs Time.

\subsection{Timing Techniques}
What timing techniques exist for synchronous and asynchronous systems? Which of these are relevant to our work?

\subsection{Patmos Time Predictable Processor}
Brief description of Patmos and why we use it.








\section{Static Verification of Artificial Neural Networks}
This section looks into ANN verification methods and why they are good/bad.

\subsection{The Problem of Artificial Neural Network Verification}
ANN verification is an issue. Why?

\subsection{Existing Techniques for Verifying ANNs}
Rundown of existing techniques of ANN verification.

\subsection{Limitations of Artificial Neural Network Verification}
Describe where verification is limited and why.
Where can it do better?





\section{Synchronous Languages}
\subsection{Synchronous Semantics}
Briefly explain synchronous semantics: what is synchronous programming and how does it work?

\subsection{Safety of Synchronous Languages}
What guarantees does synchronous programming provide?
Determinism
Causality
Easy to formalise
Etc

\subsection{Timing Correctness}
Time predictability.

\subsection{Functional Correctness}
Runtime Enforcement






\section{Run-time Enforcement}
\subsection{Run-time Assurance}
What is RA and why is it useful? Formally defined.

\subsection{Run-time Monitoring}
How does this relate and improve on RA?

\subsection{Run-time Enforcement}
Benefits? Synchronous vs asynchronous systems. 

\subsection{Safety (Timed) Automata}
These are used in our run-time enforcement techniques. Discuss these and how they work.





\section{Summary}
Summary of above.