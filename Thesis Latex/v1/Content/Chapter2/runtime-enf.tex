\section{Introduction}
\label{sec:intro2}

\acp{CPS} can be thought of as distributed networks of digital controllers that manage physical processes~\cite{alur2015principles}. 
These have strict timing and functionality requirements --- any errors in control can result in physical damage, injuries, and/or fatalities~\cite{ANNDevModel1999}. 
In the previous chapter we looked at the timing analysis of \acp{ANN}, this chapter addresses the functional requirements of \acp{ANN}.

\acp{ANN} are being increasingly used as controllers in \ac{CPS} due to their ability to learn data relationships in ways that are difficult to replicate~\cite{ANNSafety2007}. 
\acp{ANN} can deal with novel inputs to the system and are able to outperform other forms of \ac{AI} at computational efficiency, pattern recognition, function approximation and image identification~\cite{AIComp2016, AIComp2017}. 
However, it can be very difficult to ensure the safety of a system involving \acp{ANN}~\cite{ANNSafety2007, ANNSafety2018}.
%As a result, \acp{ANN} should be restricted to advisory roles in \ac{CPS}, due to the difficulty in proving the safety arguments of the \acp{ANN}.% i.e. robustness, \ac{WCET}, accuracy, etc.
%meaning they may not be able to be used to their full capabilities.

This is a concern for systems that are \textit{safety-critical}, as even though it is well-known that \acp{ANN} can be thoroughly trained and produce confidence intervals greater than 90~\%, any misbehaviour in a safety-critical system can result in  catastrophic consequences.
In order for an \ac{ANN} to be used in any capacity within such a system, it should undergo rigorous and thorough validation, verification, and testing procedures to ensure that they it is sufficiently safe for its target system~\cite{scann, ANNSafetyLifecycle2003}.
In order to have \textit{safe} \acp{ANN}, some key problems to consider are: \acp{ANN} must (1) tolerate faults and inconsistencies in their inputs, (2) not create hazardous outputs, (3) be robust and repeatable, and (4) be trained on clean, reliable data~\cite{EstSafeCriteria2003}.

To solve these problems, there exist safety measures such as risk management systems that span the entire development process of the \ac{ANN}~\cite{ANNDevModel1999} and standards with which \acp{ANN} can be certified before they are used in \acp{CPS}~\cite{SCANNStandard}, producing \acp{ANN} that are classified as \textit{safe enough} for their role. 
However, as \acp{ANN} become larger and more full-featured, they  become harder to statically analyse.
Problematic situations can arise when an \ac{ANN} exhibits unexpected behaviour that the system is unable to safely respond to.

To address this issue, we propose the use of \ac{RE} within the \ac{ANN} domain for \ac{CPS} as a methodology for protecting the safety of \ac{CPS} \textit{at runtime}.
\ac{RE} is an existing approach to ensuring the safety of \acp{CPS} and is used to enforce safety policies that monitor and edit the events of the target system~\cite{rta-cps}.
\ac{RE} is formally defined to be used in \acp{CPS}~\cite{recps}, however there has been no formal approach to combining \ac{RE} with \acp{ANN}. 

% What is in this paper
The key contributions of this chapter are thus: (1) we propose combing \acp{SNN} and \ac{RE} for the first time; (2) we provide formalisation for \ac{RE} by formally defining \acfp{VDTA} and (3) we show that this synchronous composition results in a system that is safe, i.e. it produces safe outputs and responds to real-time deadlines. 

