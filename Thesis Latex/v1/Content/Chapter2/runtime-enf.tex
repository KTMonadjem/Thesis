\section{Runtime Enforcement of Safety Critical Systems}
\label{sec:intro}

%IEEE transactions on cybernetics template.

%\acp{CPS} can be thought of as networks of digital controllers that manage physical processes~\cite{alur2015principles}. 
%Some \acp{CPS} are considered \acp{SCS}.
%These have strict timing and functionality requirements --- any errors in control can result in physical damage, injuries, and/or fatalities~\cite{ANNDevModel1999}. 
%For instance, \acp{SCS} include medical systems, autonomous vehicle controllers, and most industrial automation applications. 
%Such systems are bound by strict constraints in order to minimize failure in the system, and thus minimize the risk of danger. 

%\acp{ANN} are being increasingly used as controllers in \acp{CPS} due to their ability to learn data relationships in ways that are difficult to replicate~\cite{ANNSafety2007}. 
%\acp{ANN} can deal with novel inputs to the system and are able to outperform other forms of \ac{AI} at computational efficiency, pattern recognition, function approximation and image identification~\cite{AIComp2016, AIComp2017}. 
%However, it can be very difficult to ensure the safety of a system involving \acp{ANN}~\cite{ANNSafety2007, ANNSafety2018}.
%As a result, \acp{ANN} should be restricted to advisory roles in \acp{SCS}, % due to the difficulty in proving the safety arguments of the \acp{ANN}.% i.e. robustness, \ac{WCET}, accuracy, etc.
%meaning they may not be able to be used to their full capabilities.

\ac{ANN} modules are being increasingly used within the \acp{CPS} domain.
This is a concern for systems that are \textit{safety-critical}, as even though it is well-known that \acp{ANN} can be thoroughly trained and produce confidence intervals greater than 90~\%, any misbehaviour in a safety-critical system can result in  catastrophic consequences.
As a result, \acp{ANN} in these systems should be carefully verified to ensure that they are sufficiently safe for their target roles~\cite{scann, ANNSafetyLifecycle2003}.
%if \ac{ANN} models are used beyond a purely advisory role, i.e. if they make  making control decision, safety considerations are to be made. 
%The majority of the \acp{CPS} are safety critical, and in order for an \ac{ANN} to be used in any capacity within such a system, it should undergo rigorous and thorough validation, verification, and testing procedures to ensure that they it is sufficiently safe for its target system~\cite{scann, ANNSafetyLifecycle2003}.
%It is well known that \ac{ANN} modules work well and provide high confidence greater than 90\%, but this confidence should be 100\% where safety is concerned. 
They must be able to tolerate faults and inconsistencies in their inputs, and should be robust and determinstic on their outputs~\cite{EstSafeCriteria2003}.
%In order to have \textit{safe} \acp{ANN} in a safety critical system, some key problems to consider are that the \acp{ANN}: must tolerate faults and inconsistencies in their inputs, must not create hazardous outputs, must be robust and repeatable and must be trained on clean, reliable data~\cite{EstSafeCriteria2003}.

To solve these problems, there exist safety measures such as risk management systems that span the entire development process of the \ac{ANN}~\cite{ANNDevModel1999} and standards with which \acp{ANN} can be certified before they are used in \acp{CPS}~\cite{SCANNStandard}, producing \acp{ANN} that are classified as \textit{safe enough} for their role. 
However, as \acp{ANN} become larger and more full-featured, they  become harder to statically analyse.
Problematic situations can arise when an \ac{ANN} exhibits unexpected behaviour that the system is unable to safely respond to, and in \acp{CPS} these situations can be life threatening.

To address this issue, we propose the usage of \ac{RE} within the \ac{ANN} domain for \ac{CPS} as a methodology for protecting the safety of \ac{CPS} \textit{at runtime}.
\ac{RE} is an existing approach to ensuring the safety of \acp{CPS} and is used to enforce safety policies that monitor and edit the events of the target system~\cite{rta-cps}.
\ac{RE} is formally defined to be used in \acp{CPS}, however there has been no formal approach to combining \ac{RE} with \acp{ANN}. 

%\subsection{\acfp{SNN}}

In this chapter, we present a new definition for safe \acp{ANN}, as well as a new definition for \acfp{MNN}, termed \acfp{SNN}.
\acp{SNN} combine the previously defined \acp{MNN} in Chapter 3 with other, synchronous components, i.e. not necessarily only \acp{SANN}.
Additionally, the definition of a \ac{SANN} is redefined to encompass \acfp{CNN} in the definition of a \ac{SANN}.
Using \acp{SNN}, we can combine \ac{RE} with \acp{SANN}; this synchronous composition of \ac{RE} and \acp{ANN} provides a greatly simplified process for validating the safety of their overall systems.

% What is in this paper
The key contributions of this paper are thus: (1) we propose \acp{SNN} as a model for combing \acp{SANN} and other components, notably \ac{RE}, for the first time; (2) we provide formalisation for these \acp{SNN} by formally defining the \acp{SANN} and the \ac{RE}; and (3) we show that this synchronous composition results in a system that is safe, i.e. it produces safe outputs and responds to real-time deadlines. 

%We call this new definition \ac{SNN}.

%We aim to target the issue of un-safe \acp{ANN} by developing \acp{SNN}, which use the well known synchronous language Esterel~\cite{benveniste2003synchronous, berry2000foundations}.
%We then provide an architecture for \acp{SNN}; the synchronous %composition of \acp{SANN} and \ac{SRE}, with discrete timed automata policies that monitor the inputs and outputs of the \acp{SANN} and keep them within safe operating requirements.

