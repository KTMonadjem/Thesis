\section{Conclusions}
\label{sec:conclusion}

In this work, we have presented an approach for synchronous composition of \acf{RE} with \acfp{ANN} for \acp{CPS} by defining \acfp{SNN}.
Our \acp{SNN} demonstrate that policies specifying safe I/O behaviours for \acp{ANN} can be enforced.
These enforced policies are shown to increase the safety and efficiency of the systems within which they are placed, without adding a large overhead to the system and without decreasing the functionality of the system.
In the three case studies presented, the \ac{RE} added an average overhead of just 6.035\%, but the efficiency and safety of each system was increased by an amount that rendered the overhead negligible.

\subsection{Future Work}

In this chapter, a single enforcer was used in each benchmark.
While this was shown to be largely effective, this is not the limit to which \ac{RE} can be used in these \acp{SNN}.
In this work, we did not fully examine the enforcement of multiple networks in a meta-neural network (i.e. each network with its own enforcer).
Examining compositions of synchronous neural networks with their enforcers could lead to more capable systems that are still safe. 
Approaches such as using \ac{RE} during the training of \acp{SANN} has yet to be explored, and \ac{RE} between layers of individual \acp{SANN} has also yet to be explored.




