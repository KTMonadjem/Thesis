\section{Discussion}
\label{sec:conclusion}

In this work, we have presented an approach for synchronous composition of \acf{RE} with \acfp{SNN} for \acp{CPS}.
Our case study and benchmarks demonstrate that policies specifying safe I/O behaviours for \acp{ANN} can be enforced.
These enforced policies are shown to increase the safety and efficiency of the systems within which they are placed, without adding a large overhead to the system and without decreasing the functionality of the system.
In the three case studies presented, the \ac{RE} added an average overhead of just 6.035\%, but the efficiency and safety of each system was increased by an amount that rendered the overhead negligible.

The \ac{AV} system, when using \ac{RE} to enforce some safety policies, showed a large increase in safety.
Without the enforcer, at best the system was able to run for 73 ticks before an accident occurred, according to Table~\ref{table:avenf}.
At lower epochs of training, the system was able to run for at most 3 ticks before an accident occurred.
However, when the enforcer was implemented, the system was able to run for up to 97 ticks before an accident occurred.
At it's worst trained epoch, the system was still able to run for 58 ticks.

Table~\ref{table:avenf} also shows that the number of the accidents over all runs in the system decreased as the system was trained, even with the enforcer in place.
An untrained \ac{SNN} would have an accident on 75\% of the runs, while the same \ac{SNN} trained for 100,000 epochs would only have an accident 57\% of the time.
This shows that the enforcer, while increasing the safety of the system, does so proportionally to the level of training of the \ac{SNN} it is monitoring.

As the \ac{SNN} was trained, it exhibited more careful behaviour; the vehicle's average speed decreased at and the vehicle performed more braking actions than necessary.
Even with the enforcer in place, the speed decreased in proportion to the number of epochs trained, while the vehicle still braked unnecessarily.

While the enforcer allowed the vehicle to drive safely for longer periods of time, it did not decrease the number of incidents in the system proportionally to the time the vehicle stayed safe.
Likewise, the enforcer did not result in the vehicle driving at a reasonable speed.
With or without the enforcer, the vehicle braked more often and drove more slowly.
We can conclude that the enforcer did efficiently increase the safety of the system by our given measurement, however other safety aspects of the system were not addressed to the same level.
This shows that the amount of training the \ac{SNN} has is relevant, enforcer or no.

The \ac{ESS} (Table~\ref{table:essres}) showed that the enforcer increased safety by preventing dangerously low \acf{SoC}.
However, with RABBIT game (Table~\ref{table:rabbitres}) the enforcer increased the efficiency of the players in the game, as the game had no safety requirements.







This is a concern for systems that are \textit{safety-critical}, as even though it is well-known that \acp{ANN} can be thoroughly trained and produce confidence intervals greater than 90~\%, any misbehaviour in a safety-critical system can result in  catastrophic consequences.
In order for an \ac{ANN} to be used in any capacity within such a system, it should undergo rigorous and thorough validation, verification, and testing procedures to ensure that they is sufficiently safe for its target system~\cite{scann, ANNSafetyLifecycle2003}.
In order to have \textit{safe} \acp{ANN}, some key problems to consider are: \acp{ANN} must (1) tolerate faults and inconsistencies in their inputs; (2) not create hazardous outputs; (3) be robust and repeatable; and (4) be trained on clean, reliable data~\cite{EstSafeCriteria2003}.

To solve these problems, there exist safety measures such as risk management systems that span the entire development process of the \ac{ANN}~\cite{ANNDevModel1999} and standards with which \acp{ANN} can be certified before they are used in \acp{CPS}~\cite{SCANNStandard}, producing \acp{ANN} that are classified as \textit{safe enough} for their role. 
However, as \acp{ANN} become larger and more full-featured, they  become harder to statically analyse.
Problematic situations can arise when an \ac{ANN} exhibits unexpected behaviour that the system is unable to safely respond to.

To address this issue, we propose the use of \ac{RE} within the \ac{ANN} domain for \ac{CPS} as a methodology for protecting the safety of \ac{CPS} \textit{at runtime}.
\ac{RE} is an existing approach to ensuring the safety of \acp{CPS} and is used to enforce safety policies that monitor and edit the events of the target system~\cite{rta-cps}.
\ac{RE} is formally defined to be used in \acp{CPS}~\cite{recps}, however there has been no formal approach to combining \ac{RE} with \acp{ANN}. 

% What is in this paper
The key contributions of this chapter are thus: (1) combining \acp{SNN} and \ac{RE} for the first time; (2) formalisation for \ac{RE} of \acp{SNN} by formally defining \acfp{VDTA}; and (3) demonstrating that this synchronous composition results in a system that is safe, i.e. it produces safe outputs and responds to real-time deadlines. 





