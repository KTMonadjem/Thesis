\section{Cyber-Physical Systems}
Introduce the concept of safety critical, CPS. What makes a system safety critical?

\section{Artificial Neural Networks}
Introduce AI, more specifically ANNs. What they can do?

\section{Contribution}
Creating methods of implementing ANNs in safety critical systems using synchronous semantics.

\section{Thesis Structure}
Chapter 2 gives a basic understanding of the concepts required to understand this research towards safe \acp{ANN} for \ac{CPS}.

Chapter 3 introduces the concept of \acfp{SANN} and their timing properties.
Formal definitions of \acp{SANN} and their related components are provided in this chapter.
Furthermore, the combinations of these \acp{SANN} and the usefulness of such in \ac{CPS} is discussed.
A new compiler to create these \acp{SANN} from Python code is also introduced.

Chapter 4 introduces the concept of \acf{RE} in combination with \acp{SANN}. 
This chapter also provides formal definitions for the combination of \acp{SANN} and \ac{RE} by expanding the definitions introduced in Chapter 3.
An \acf{AV} case study is made for this chapter.

Chapter 5 provides two different methods, used in tandem, to increase the safety of systems with complex inputs, such as object detection for \acp{AV}.
The first method is the use of \acp{MNN} to increase the classification accuracy of \acp{SANN}.
The second expands on Chapter 5, and introduces the use of \acf{RV} to increase the safety of system where \ac{RE} is impossible.
Results are presented using another \ac{AV} system case study.

