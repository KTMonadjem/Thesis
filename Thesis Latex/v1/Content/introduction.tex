\section{Cyber-Physical Systems}
\acp{CPS} use a set of controllers that are distributed across a network for the 
control of physical processes~\cite{alur2015principles}. \ac{CPS} applications encompass real-time systems, where 
the system needs to satisfy a set of timing requirements to ensure correct operation. Examples include autonomous vehicles and
smart power grids. Here, a missed deadline may result in catastrophic consequences, making these \acp{CPS} highly 
\textit{safety-critical}. 
These have strict timing and functionality requirements --- any errors in control can result in physical damage, injuries, and/or fatalities~\cite{ANNDevModel1999}. 

\section{Artificial Neural Networks}


\section{Contribution}
This thesis provides novel techniques to the verification and validation of \acfp{ANN} in \acfp{CPS}.
There is a large body of work regarding the safety of \acp{ANN}, ranging across a whole variety of applications and approaches.
However, no work has been done regarding the synchronous implementation of \acp{ANN} and the verification available to synchronous \acp{ANN}.  
This thesis addresses the issue of safe \acp{ANN} using synchronous semantics.

In this thesis, a new \ac{ANN} library was created to look at the benefits of synchronous \acp{ANN}.
In some of the benchmarks, an existing library called Darknet~\cite{darknet13} was used to implement some of the more complex \acp{ANN}.
However, a tool chain was created to replace Darknet that compiles Keras~\cite{chollet2015keras} \acp{ANN} to the created \ac{ANN} library.

The major contributions of this thesis are as follows:
\begin{itemize}
	\item A time predictable approach to \acp{ANN} is developed using the synchronous language Esterel~\cite{berry2000foundations}. These predictable \acp{ANN} are termed \acfp{SANN} and are defined using formal methods. Using T-CREST as a platform, timing of these \acp{SANN} was done. These \acp{SANN} provide a safe approach to implementing \acp{ANN} in \acp{CPS}. The results of the \acp{SANN} are presented using a set of benchmarks. 
	\item \acfp{MNN} are proposed as a framework for the composition of multiple \acp{SANN}. The thesis provides formal definitions for the \acp{MNN} and benchmarks are presented that show their efficacy. 
	\item The thesis proposes the \acf{RV} of \acp{MNN} as an approach to dealing with input perturbation of \acp{ANN} that have complex inputs, such as image classification \acfp{CNN}. An \acf{AV} case study is created for this work, with an \ac{AV} object detection simulation created to prove the benefits of this proposition. 
\end{itemize}

\section{Thesis Structure}
Chapter 2 gives a basic understanding of the concepts required to understand this research towards safe \acp{ANN} for \ac{CPS}.

Chapter 3 introduces the concept of \acfp{SANN} and their timing properties.
Formal definitions of \acp{SANN} and their related components are provided in this chapter.
Furthermore, the meta combinations of these \acp{SANN}, termed \acp{MNN}, and the usefulness of such \acp{MNN} in \ac{CPS} is discussed.
Lastly, a new Python tool chain that creates these \acp{SANN} from Keras is also introduced.

Chapter 4 introduces the concept of \acf{RE} in combination with \acp{SANN}. 
This chapter provides formal definitions for the combination of \acp{SANN} and \ac{RE} by expanding the definitions of \acp{MNN} introduced in Chapter 3 to include other synchronous components besides \acp{SANN}.
An \acf{AV} case study is made for this chapter to show the efficacy of the \ac{RE} of \acp{SANN}.

Chapter 5 proposes two different methods, used in tandem, to increase the safety of systems with complex inputs, such as object detection for \acp{AV}.
The first method is the use of \acp{MNN} to increase the classification accuracy of \acp{SANN}.
The second expands on Chapter 4, and introduces the use of \ac{RV} to increase the safety of system where \ac{RE} cannot do so.
An \ac{AV} object detection simulation is created as a complex \ac{MNN} 

Finally, conclusions are drawn on the synchronous approach proposed in this thesis to creating safe \acp{ANN}.

